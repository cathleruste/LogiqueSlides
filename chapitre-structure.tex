\section{Les variables structurées}
%\leconwithtoc

\subsection{Le type structuré}

\begin{frame}{types simple et structuré}
	\begin{itemize}
		\item
		Les variables de \code{types}
		dits «~\code{simples}~» ne peuvent
		contenir qu’une seule valeur à la fois  
		(entier, réel, booléen, caractère, chaine).
		
		\bigskip
	
		\item
		Cependant, certains types
		d’information consistent en un 
		regroupement de plusieurs données
		élémentaires.
		\begin{itemize}
			\item 
				une \textbf{date} est composée de trois éléments (le jour, le mois,
				l’année);
			\item {
				un \textbf{moment} de la journée est un triple d’entiers (heures,
				minutes, secondes)}
			\item {
				la localisation d’un \textbf{point} dans un plan nécessite 
				deux coordonnées cartésiennes (l’abscisse \textit{x} et
				l’ordonnée \textit{y}) ou polaires (le rayon \textit{r} et l’angle
				\textit{$\theta $})}
			\item {
				une \textbf{adresse }est composée de plusieurs données}
		\end{itemize}
		
	\end{itemize}
\end{frame}

\begin{frame}{type structuré}
	Pour stocker et manipuler de tels ensembles de données, nous utiliserons
	des \textbf{types structurés} ou \textbf{structures}.
	
	\bigskip
	
	\cadre{
	Une \code{structure} est donc un ensemble
	fini d’éléments pouvant être de types distincts. Chaque élément de cet
	ensemble, appelé \code{champ} de la structure, possède un nom unique.
	}
	
	\bigskip
	
	Noter qu’un champ d’une structure peut lui-même être une structure. 
	
	Par	exemple, une \textbf{carte d’identité} inclut parmi ses informations
	une \textbf{date} de naissance, l’\textbf{adresse} de son
	propriétaire (cachée dans la puce électronique!),\dots
\end{frame}

\subsection{Définition d'une structure}

\begin{frame}{Définition}
	La définition d’un type structuré adoptera le modèle suivant~:

	\bigskip
	
	\cadre{
	\begin{pseudo}
	\Struct{NomDeLaStructure}
		\Decl nomChamp1~: type1
		\Decl nomChamp2~: type2
		\Decl \dots
		\Decl nomChampN~: typeN
	\EndStruct
	\end{pseudo}
	}

	\bigskip
	
	\textit{nomChamp1}, \dots, \textit{nomChampN }
	sont les noms des différents champs de la structure, 
	et \textit{type1}, \dots, \textit{typeN} 
	les types de ces champs. 
	
	\bigskip
	
	Ces types sont 
	\begin{itemize}
	\item
	soit les types «~simples~» étudiés
	précédemment (entier, réel, booléen, caractère, chaine) 
	\item
	soit d’autres
	types structurés dont la structure aura été préalablement définie.
	\end{itemize}
\end{frame}

\begin{frame}{Exemple}
	Pour exemple, nous définissons ci-dessous trois
	types structurés que nous utiliserons souvent par la suite~:

	\bigskip
	
	\cadre{
	\begin{pseudo}
	\Struct{Date}
		\Decl jour~: entier
		\Decl mois~: entier
		\Decl année~: entier
	\EndStruct
	\end{pseudo}
	}
\end{frame}

\begin{frame}{Exemple}
	\cadre{
	\begin{pseudo}
	\Struct{Moment}
		\Decl heure~: entier
		\Decl minute~: entier
		\Decl seconde~: entier
	\EndStruct
	\end{pseudo}
	}

	\bigskip
	
	\cadre{
	\begin{pseudo}
	\Struct{Point}
		\Decl x~: réel
		\Decl y~: réel
	\EndStruct
	\end{pseudo}
	}

\end{frame}

\subsection{Déclaration d’une variable de type structuré}

\begin{frame}{Exemple}
	Une fois un type structuré défini, la
	déclaration de variables de ce type est identique à celle des variables
	simples. 
	
	\bigskip
	
	Par exemple~:

	\bigskip
	
	\cadre{
	\begin{pseudo}
	\Decl anniversaire, jourJ~: Date
	\Decl départ, arrivée, unMoment~: Moment
	\Decl a, b, centreGravité~: Point
	\end{pseudo}
	}
\end{frame}

\subsection{Utilisation des variables de type structuré}

\begin{frame}{Utilisation d'un champ}
	Pour accéder à l’un des champs d’une variable structurée, il faut mentionner
	le nom de ce champ ainsi que celui de la variable dont il fait partie.
	
	\bigskip
	
	Nous utiliserons la notation «~pointée~»~:
	
	\bigskip

	\cadre{
	\begin{pseudo}
	\Stmt nomVariable.nomChamp
	\end{pseudo}
	}
\end{frame}

\begin{frame}{Exemple}	
	\cadre{
	\begin{pseudo}
	\Let anniversaire.jour \Gets 15
	\Let anniversaire.mois \Gets 10
	\Let anniversaire.année \Gets 2014
	\Let arrivée.heure \Gets départ.heure + 2
	\Let centreGravité.x \Gets (a.x + b.x) / 2
	\end{pseudo}
	}
\end{frame}

\begin{frame}{Utilisation globale}
		On peut cependant, dans certains cas, utiliser
	une variable structurée de manière globale (c’est-à-dire d’une façon
	qui agit simultanément sur chacun de ses champs). 
	
	\bigskip
	
	Le cas le plus
	courant est l’affectation interne entre deux variables structurées de
	même type, par exemple~:

	\bigskip
	
	\cadre{
	\begin{pseudo}
	\Let arrivée \Gets départ
	\end{pseudo}
	}

	\bigskip
	
	qui résume les trois instructions suivantes~:

	\bigskip
	
	\cadre{
	\begin{pseudo}
	\Let arrivée.heure \Gets départ.heure
	\Let arrivée.minute \Gets départ.minute
	\Let arrivée.seconde \Gets départ.seconde
	\end{pseudo}
	}
\end{frame}

\begin{frame}{Paramètre d'un module}
	Une variable structurée peut aussi être le
	paramètre d’un module, et un module peut également renvoyer une
	«~valeur~» de type structuré. 
	
	\bigskip
	
	Par exemple, l’entête d’un module
	renvoyant le nombre de secondes écoulées entre une heure de départ et
	d’arrivée serait~:

	\bigskip
	
	\cadre{
	\begin{pseudo}
	\ModuleSign{nbSecondesEcoulées}{ départ\In, arrivée\In~: Moment}{entier}
	\end{pseudo}
	}
\end{frame}

\begin{frame}{Lire et afficher}
	On pourra aussi lire ou afficher une variable structurée.

	\bigskip
	
	\cadre{
	\begin{pseudo}
	\Read unMoment
	\Write unMoment
	\end{pseudo}
	}
\end{frame}

\begin{frame}{Comparaison}
	On ne peut pas utiliser les opérateurs de comparaison ({\textless}, {\textgreater}) pour
	comparer des variables structurées. 
	
	\bigskip
	
	En effet, s’il est naturel de vouloir comparer des dates ou des moments,
	comment définir une relation d’ordre avec les points du plan ou avec
	des cartes d’identités ?

	\bigskip
	
	Si le besoin de comparer des variables
	structurées se fait sentir, il faudra dans ce cas écrire des modules de
	comparaison adaptés aux structures utilisées.
\end{frame}

\begin{frame}{Assignation globale}
	Par facilité d'écriture, on peut assigner tous les champs en une 
	fois via des «~\{\}~». 
	
	\bigskip
	
	Exemple~:

	\bigskip
	
	\cadre{
	\begin{pseudo}
	\Let anniversaire \Gets \{1, 9, 1989\}
	\end{pseudo}
	}
\end{frame}

\subsection{Exemple d’algorithme}

\begin{frame}{Exemple}	
	Le module ci-dessous reçoit en paramètre deux dates ; 
	
	la valeur renvoyée est 
	\begin{itemize}
	\item
	–1 si la première date est antérieure à la deuxième, 
	\item
	0 si les deux	dates sont identiques 
	\item
	1 si la première date est postérieure à la
	deuxième.
	\end{itemize}
\end{frame}

\begin{frame}{Exemple}	
		\cadre{
	\begin{pseudo}
	\Module{comparerDates}{date1\In, date2\In~: Date}{entier}
		\Decl résultat~: entier
		\Let  résultat \Gets -1
		\RComment valeur choisie par défaut
		\If{date1.année ${\geq}$ date2.année}
			\If{date1.année $>$ date2.année}
				\Let  résultat \Gets 1
			\Else \RComment les années sont identiques
				\If{date1.mois ${\geq}$ date2.mois}
					\If{date1.mois $>$ date2.mois}
						\Let  résultat \Gets 1
					\Else \RComment les années et les mois sont identiques
						\If{date1.jour ${\geq}$ date2.jour}
							\If{date1.jour $>$ date2.jour}
								\Let  résultat \Gets 1
							\Else \RComment les années, les mois et les jours sont identiques
								\Let  résultat \Gets 0
							\EndIf
						\EndIf
					\EndIf
				\EndIf
			\EndIf
		\EndIf
		\Return résultat
	\EndModule
	\end{pseudo}
	}
\end{frame}
