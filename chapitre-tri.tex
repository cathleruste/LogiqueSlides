\section{Les tris}
%\leconwithtoc

\subsection{Objectif}

\begin{frame}
	Voyons quelques algorithmes simples pour trier un
	ensemble d'informations~: 
	
	\begin{itemize}
		\item
		recherche des maxima, 
		\item
		tri	par insertion et 
		\item
		tri bulle 
	\end{itemize}
	dans un tableau. 
	
	\bigskip
	
	Des algorithmes plus efficaces seront vus
	en deuxième année.
\end{frame}

\subsection{Motivation}

\begin{frame}{Motivation~: tri d'informations}
	\begin{itemize}
		\item
		recherche d’un numéro de téléphone ou une adresse
		dans la version informatisée des pages blanches, 
		\item
		efficacité de certains moteurs de recherche qui
		permettent de retrouver en quelques secondes des mots clés arbitraires
		parmi plusieurs millions de pages sur le web.
	\end{itemize}
\end{frame}

\begin{frame}{Motivation}
	\begin{itemize}
		\item
		La recherche efficace d’information implique un tri préalable de
		celle-ci. 
		\item
		En effet, si les données ne sont pas classées ou triées, le
		seul algorithme possible reviendrait à parcourir entièrement l’ensemble
		des informations. 
		\item
		Il suffit d’imaginer un dictionnaire
		dans lequel les mots seraient mélangés de façon aléatoire au lieu
		d’être classés par ordre alphabétique. Pour trouver le moindre mot dans
		ce dictionnaire, il faudrait à chaque fois le parcourir entièrement !
		\item
		Il est clair que le classement préalable (ordre alphabétique) accélère
		grandement la recherche.
	\end{itemize}
\end{frame}

\begin{frame}{Motivation}
	Ainsi, 
	\begin{itemize}
		\item
		recherche et tri sont étroitement liés, 
		\item
		la façon dont les informations sont triées conditionne 
		bien entendu la façon de rechercher l’information 
		(cf. algorithme de recherche dichotomique).
	\end{itemize}
	
	Pour exemple, prenons cette fois-ci un dictionnaire des mots croisés
	dans lequel les mots sont d’abord regroupés selon leur longueur et
	ensuite par ordre alphabétique. 
	
	La façon de rechercher un mot dans ce
	dictionnaire est bien sûr différente de la recherche dans un
	dictionnaire usuel. 
\end{frame}

\begin{frame}{Motivation}
	Le problème central est donc le tri des informations. 
	
	\bigskip
	
	Celui-ci a pour	but d’organiser un ensemble 
	d’informations qui ne l’est pas \textit{à
	priori}. 
	
	\bigskip
	
	On peut distinguer trois grands cas de figure~:
\end{frame}

\begin{frame}{Motivation}
	\begin{itemize}
		\item 
			D’abord les situations impliquant le classement total d’un ensemble de
			données «~brutes~», c’est-à-dire complètement désordonnées. 
			
			Prenons	pour exemple les feuilles récoltées en vrac à l’issue 
			d’un examen ; il y a peu de chances que celles-ci soient remises 
			à l’examinateur de	manière ordonnée ; 
			
			celui-ci devra donc procéder au tri de l’ensemble
			des copies, par exemple par ordre alphabétique des noms des étudiants,
			ou par numéro de groupe etc.
	\end{itemize}
\end{frame}
	
\begin{frame}{Motivation}
	\begin{itemize}
		\item 
			Ensuite les situations où on s’arrange pour ne jamais devoir trier la
			totalité des éléments d’un ensemble, qui resterait cependant à tout
			moment ordonné. 
			
			Imaginons le cas d’une bibliothèque dont les livres
			sont rangés par ordre alphabétique des auteurs~: 
			\begin{itemize}
				\item
				à l’achat d’un nouveau livre, ou au retour de prêt d’un livre, 
				celui-ci est immédiatement rangé à la bonne place. 
				\item
				Ainsi, l’ordre global de la bibliothèque est
				maintenu par la répétition d’une seule opération élémentaire consistant
				à insérer à la bonne place un livre parmi la collection. 
			\end{itemize}
			C’est la	situation que nous considérerions dans le cas d'une structure
			où les éléments sont ordonnés.
	\end{itemize}
\end{frame}
	
\begin{frame}{Motivation}
	\begin{itemize}
		\item 
			Enfin, les situations qui consistent à devoir re-trier des données
			préalablement ordonnées sur un autre critère. 
			
			Prenons l’exemple 
			\begin{itemize}
				\item
				d’un paquet de copies d’examen déjà triées sur l’ordre alphabétique des noms
				des étudiants, 
				\item
				et qu’on veut re-trier cette fois-ci sur les numéros de
				groupe. 
			\end{itemize}
			Il est clair qu’une méthode efficace veillera à conserver
			l’ordre alphabétique déjà présent dans la première situation afin que
			les copies apparaissent dans cet ordre dans chacun des groupes.
	\end{itemize}
\end{frame}
	
\begin{frame}{Motivation}
	Le dernier cas illustre un classement sur une \textbf{clé complexe}
	(ou \textbf{composée}) impliquant la comparaison de plusieurs champs
	d’une même structure~: 
	
	le premier classement se fait sur le numéro de
	groupe, et à numéro de groupe égal, l’ordre se départage sur le nom de
	l’étudiant. 
	
	On dira de cet ensemble qu’il est classé en \textbf{majeur}
	sur le numéro de groupe et en \textbf{mineur} sur le nom d’étudiant.
\end{frame}

\begin{frame}{Motivation}
	Notons que certains tris sont dits \textbf{stables} parce
	qu'en cas de tri sur une nouvelle clé, l’ordre de la
	clé précédente est préservé pour des valeurs identiques de la nouvelle
	clé, ce qui évite de faire des comparaisons sur les deux champs à la
	fois. 
	
	\bigskip
	
	Les méthodes nommées \textbf{tri par insertion}, \textbf{tri
	bulle} et \textbf{tri par recherche de minima successifs }(que nous
	allons aborder dans ce chapitre) sont stables.
\end{frame}

\begin{frame}{Motivation}
	Le tri d’un ensemble d’informations n’offre que des avantages. 
	
	\begin{itemize}
		\item
		Outre le fait déjà mentionné de permettre une recherche et une obtention rapide
		de l’information, 
		\item
		il permet aussi l’application de traitements
		algorithmiques efficaces (comme par exemple celui des ruptures que nous
		verrons plus loin) qui s’avéreraient trop coûteux (en temps, donc en
		argent !) s’ils s’effectuaient sur des ensembles non triés.
	\end{itemize}
\end{frame}

\begin{frame}{Motivation}
	Certains tris sont évidemment plus performants que d’autres. 
	
	\bigskip
	
	Le choix
	d’un tri particulier dépend de la taille de l’ensemble à trier et de la
	manière dont il se présente, c’est-à-dire déjà plus ou moins ordonné.
	
	\bigskip
	
	La performance quant à elle se juge sur deux critères~: 
	\begin{itemize}
		\item
		le nombre de tests effectués (comparaisons de valeurs) et 
		\item
		le nombre de transferts de valeurs réalisés. 
	\end{itemize}
\end{frame}

\begin{frame}{Motivation}
	Dans ce chapitre, les algorithmes traiteront du tri dans un
	\textbf{tableau d’entiers à une dimension}. 
	
	\bigskip
	
	Toute autre	situation peut bien entendu se ramener à celle-ci moyennant la
	définition de la relation d’ordre propre au type de données utilisé. 
	
	\bigskip
	
	Ce sera par exemple, l’ordre alphabétique pour les chaines de caractères,
	l’ordre chronologique pour des objets \textit{Date} ou
	\textit{Moment} (que nous verrons
	plus tard), etc.
	
	\bigskip
	
	De plus, le seul ordre envisagé sera l’ordre
	\textbf{croissant} des données. 
	
	\bigskip
	
	Plus loin, on envisagera le tri 
	d'autres structures de données.
\end{frame}

\begin{frame}{Motivation}
	Enfin, dans toutes les méthodes de tri abordées, nous supposerons la
	\textbf{taille physique} du tableau à trier (notée $n$) \textbf{égale} 
	à sa \textbf{taille	logique}, 
	celle-ci n’étant pas modifiée par l’action de tri.
\end{frame}

\subsection{Tri par insertion}

\begin{frame}{Tri par insertion}
	Cette méthode de tri repose sur le principe d’insertion de valeurs dans
	un tableau ordonné. 
\end{frame}

\begin{frame}{Tri par insertion}
	Le tableau à trier sera à chaque étape subdivisé en deux sous-tableaux~:
	\begin{itemize}
		\item
		le premier cadré à gauche contiendra des éléments déjà ordonnés, 
		\item
		et le	second, cadré à droite, ceux qu’il reste à insérer dans 
		le sous-tableau	trié. 
	\end{itemize}
	
	Le premier verra sa taille s’accroitre au fur et à mesure des
	insertions, 
	
	tandis que celle du sous-tableau des éléments non triés
	diminuera progressivement.
\end{frame}

\begin{frame}{Tri par insertion}
	\begin{itemize}
		\item
		Au départ de l’algorithme, le sous-tableau trié est le premier élément
		du tableau. 
		
		Comme il ne possède qu’un seul élément, ce sous-tableau est
		donc bien ordonné ! 
		\item
		Chaque étape consiste ensuite à prendre le premier
		élément du sous-tableau non trié et à l’insérer à la bonne place dans
		le sous-tableau trié.
	\end{itemize}
\end{frame}

\begin{frame}{Tri par insertion}
	Prenons comme exemple un tableau \code{tab} de 20 entiers. 
	
	Au départ, le	sous-tableau trié est formé du premier élément, 
	\code{tab[1]} qui vaut 20~:
	\begin{center}
	\begin{tabular}{*{20}{>{\centering\sffamily\itshape\arraybackslash}m{0.4cm}}}
		 1 &
		 2 &
		 3 &
		 4 &
		 5 &
		 6 &
		 7 &
		 8 &
		 9 &
		 10 &
		 11 &
		 12 &
		 13 &
		 14 &
		 15 & 
		 16 &
		 17 &
		 18 &
		 19 &
		 20
		 \\
	\end{tabular}
	\begin{supertabular}{|*{20}{>{\centering\arraybackslash}m{0.4cm}|}}
		\hline
		\multicolumn{1}{|m{0.4cm}|}{\cellcolor{gray!25}20} &
		{12} &
		{18} &
		{17} &
		{15} &
		{14} &
		{15} &
		{16} &
		{18} &
		{17} &
		{12} &
		{14} &
		{16} &
		{18} &
		{15} &
		{15} &
		{9} &
		{11} &
		{11} &
		{13}\\\hline
	\end{supertabular}
	\end{center}

\end{frame}

\begin{frame}{Tri par insertion}
	L’étape suivante consiste à insérer \code{tab[2]}, qui vaut 12, dans ce sous-tableau, de
	taille 2~:

	\begin{center}
	\begin{tabular}{*{20}{>{\centering\sffamily\itshape\arraybackslash}m{0.4cm}}}
		 1 &
		 2 &
		 3 &
		 4 &
		 5 &
		 6 &
		 7 &
		 8 &
		 9 &
		 10 &
		 11 &
		 12 &
		 13 &
		 14 &
		 15 & 
		 16 &
		 17 &
		 18 &
		 19 &
		 20
		 \\
	\end{tabular}
	\begin{tabular}{|*{20}{>{\centering\arraybackslash}m{0.4cm}|}}
		\hline
		\multicolumn{1}{|m{0.4cm}|}{\cellcolor{gray!25}12} &
		{\cellcolor{gray!25}20} &
		{18} &
		{17} &
		{15} &
		{14} &
		{15} &
		{16} &
		{18} &
		{17} &
		{12} &
		{4} &
		{16} &
		{18} &
		{15} &
		{5} &
		{19} &
		{11} &
		{11} &
		{13}\\\hline
	\end{tabular}
	\end{center}
\end{frame}

\begin{frame}{Tri par insertion}
	Ensuite, c’est au tour de \code{tab[3]}, qui vaut 18, d’être inséré~:

	\begin{center}
	\begin{tabular}{*{20}{>{\centering\sffamily\itshape\arraybackslash}m{0.4cm}}}
		 1 &
		 2 &
		 3 &
		 4 &
		 5 &
		 6 &
		 7 &
		 8 &
		 9 &
		 10 &
		 11 &
		 12 &
		 13 &
		 14 &
		 15 & 
		 16 &
		 17 &
		 18 &
		 19 &
		 20
		 \\
	\end{tabular}
	\begin{tabular}{|*{20}{>{\centering\arraybackslash}m{0.4cm}|}}
		\hline
		{\cellcolor{gray!25}12} &
		{\cellcolor{gray!25}18} &
		{\cellcolor{gray!25}20} &
		{17} &
		{15} &
		{14} &
		{15} &
		{6} &
		{18} &
		{17} &
		{12} &
		{14} &
		{16} &
		{18} &
		{15} &
		{15} &
		{9} &
		{11} &
		{11} &
		{13}\\\hline
	\end{tabular}
	\end{center}
	
	Et ainsi de suite jusqu’à insertion du dernier élément dans le
	sous-tableau trié. 

\end{frame}

\begin{frame}{Tri par insertion~: algorithme}
	On combine la recherche de la position d’insertion et le décalage
	concomitant de valeurs.

	\cadre{
	\begin{pseudo}
		\LComment Trie le tableau reçu en paramètre (via un tri par insertion).
		\Module{triInsertion}{tab\InOut~: tableau[1 à n] d’entiers}{}
			\Decl i, j, valAInsérer~: entiers
			\For{i \K{de} 2 \K{à} n}
				\Let valAInsérer \Gets tab[i]
				\LComment recherche de l’endroit où insérer valAInsérer dans le 
				\LComment sous-tableau trié et décalage simultané des éléments
				\Let j \Gets i - 1
				\While{j ${\geq}$ 1 ET valAInsérer < tab[j]}
					\Let tab[j+1] \Gets tab[j]
					\Let j \Gets j – 1
				\EndWhile
				\Let tab[j+1] \Gets valAInsérer
			\EndFor
		\EndModule
	\end{pseudo}
	}

\end{frame}

\subsection{Tri par sélection des minima successifs}

\begin{frame}{Tri par sélection des minima successifs}
	Dans ce tri, on recherche à chaque étape la plus petite valeur de
	l’ensemble non encore trié et on peut la placer immédiatement 
	à sa position	définitive.
\end{frame}

\begin{frame}{Tri par sélection des minima successifs}
	Prenons par exemple un tableau de 20 entiers. 
	
	La première étape consiste
	à rechercher la valeur minimale du tableau. Il s’agit de l’élément
	d’indice 10 et de valeur 17.
	
	\begin{center}
	\begin{tabular}{*{20}{>{\centering\sffamily\itshape\arraybackslash}m{0.4cm}}}
		 1 &
		 2 &
		 3 &
		 4 &
		 5 &
		 6 &
		 7 &
		 8 &
		 9 &
		 10 &
		 11 &
		 12 &
		 13 &
		 14 &
		 15 & 
		 16 &
		 17 &
		 18 &
		 19 &
		 20
		 \\
	\end{tabular}
	\begin{tabular}{|*{20}{>{\centering\arraybackslash}m{0.4cm}|}}
		\hline
		{20} &
		{52} &
		{61} &
		{47} &
		{82} &
		{64} &
		{95} &
		{66} &
		{84} &
		{\cellcolor{gray!25}17} &
		{32} &
		{24} &
		{46} &
		{48} &
		{75} &
		{55} &
		{19} &
		{61} &
		{21} &
		{30}\\\hline
	\end{tabular}
	\end{center}
\end{frame}

\begin{frame}{Tri par sélection des minima successifs}
	Celui-ci devrait donc apparaitre en 1\textsuperscript{ère }
	position du	tableau. 
	
	Hors, cette position est occupée par la valeur 20. 
	
	\bigskip
	
	Comment	procéder dès lors pour ne perdre aucune valeur 
	et sans faire appel à un second tableau ? 
	
	\pause
	
	Il suffit d’échanger le
	contenu des deux éléments d’indices 1 et 10~:
\end{frame}

\begin{frame}{Tri par sélection des minima successifs}
		\begin{center}
	\begin{tabular}{*{20}{>{\centering\sffamily\itshape\arraybackslash}m{0.4cm}}}
		 1 &
		 2 &
		 3 &
		 4 &
		 5 &
		 6 &
		 7 &
		 8 &
		 9 &
		 10 &
		 11 &
		 12 &
		 13 &
		 14 &
		 15 & 
		 16 &
		 17 &
		 18 &
		 19 &
		 20
		 \\
	\end{tabular}
	\begin{tabular}{|*{20}{>{\centering\arraybackslash}m{0.4cm}|}}
		\hline
		{\cellcolor{gray!25}17} &
		{52} &
		{61} &
		{47} &
		{82} &
		{64} &
		{95} &
		{66} &
		{84} &
		{20} &
		{32} &
		{24} &
		{46} &
		{48} &
		{75} &
		{55} &
		{19} &
		{61} &
		{21} &
		{30}\\\hline
	\end{tabular}
	\end{center}
\end{frame}

\begin{frame}{Tri par sélection des minima successifs}
	\begin{itemize}
		\item
		Le tableau se subdivise à présent en deux sous-tableaux, 
		\begin{itemize}
			\item
			un sous-tableau	déjà trié (pour le moment réduit au seul élément $tab[1]$) 
			\item 
			et le sous-tableau des autres valeurs non encore triées (de l’indice 2 à 20).
		\end{itemize}
		\item
		On recommence ce processus dans ce second sous-tableau~: le minimum est
		à présent l'élément d’indice 17 et de valeur 19.
		
		Celui-ci viendra donc en 2\textsuperscript{ème} position, échangeant sa
		place avec la valeur 52 qui s’y trouvait~:
	\end{itemize}
\end{frame}

\begin{frame}{Tri par sélection des minima successifs}
	\begin{center}
	\begin{tabular}{*{20}{>{\centering\sffamily\itshape\arraybackslash}m{0.4cm}}}
		 1 &
		 2 &
		 3 &
		 4 &
		 5 &
		 6 &
		 7 &
		 8 &
		 9 &
		 10 &
		 11 &
		 12 &
		 13 &
		 14 &
		 15 & 
		 16 &
		 17 &
		 18 &
		 19 &
		 20
		 \\
	\end{tabular}
	\begin{tabular}{|*{20}{>{\centering\arraybackslash}m{0.4cm}|}}
		\hline
		{\cellcolor{gray!25}17} &
		{\cellcolor{gray!25}19} &
		{61} &
		{47} &
		{82} &
		{64} &
		{95} &
		{66} &
		{84} &
		{20} &
		{32} &
		{24} &
		{46} &
		{48} &
		{75} &
		{55} &
		{52} &
		{61} &
		{21} &
		{30}\\\hline
	\end{tabular}
	\end{center}
	
	\bigskip
	
	Le sous-tableau trié est maintenant formé des deux premiers éléments, et
	le sous-tableau non trié par les 18 éléments suivants. 

\end{frame}

\begin{frame}{Tri par sélection des minima successifs}
	\begin{itemize}
		\item
		Pour un tableau de taille $n$, 
		à l’étape $i$ de cet algorithme, on sélectionne donc le minimum du
		sous-tableau non trié (entre les indices $i$ et $n$). 
		\item
		Une	fois le minimum localisé, on l’échange avec l’élément d’indice
		$i$. 
		\item
		À chaque étape, la taille du sous-tableau trié augmente de
		1 et celle du sous-tableau non trié diminue de 1. 
		\item
		L’algorithme s’arrête
		à la $n$\textsuperscript{ème} étape, lorsque le sous-tableau
		trié correspond au tableau de départ. 
		\item
		En pratique l’arrêt se fait après
		l’étape $n - 1$, car lorsque le sous-tableau non trié n’est plus
		que de taille 1, il contient nécessairement le maximum de l’ensemble.
	\end{itemize}
\end{frame}

\begin{frame}{Tri par sélection des minima successifs~: algorithme}
	\cadre{
	\begin{pseudo}
		\LComment Trie le tableau reçu en paramètre (via un tri par sélection des minima successifs).
		\Module{triSélectionMinimaSuccessifs}{tab\InOut~: tableau[1 à n] d’entiers}{}
			\Decl i, indiceMin~: entier
			\For{i \K{de} 1 \K{à} n – 1}
			\RComment i correspond à l’étape de l’algorithme
				\Let indiceMin \Gets positionMin( tab, i, n )
				\Stmt swap( tab[i], tab[indiceMin] )
			\EndFor
		\EndModule
	\end{pseudo}
	}
\end{frame}

\begin{frame}{Tri par sélection des minima successifs~: algorithme}
	\cadre{
	\begin{pseudo}
		\LComment Retourne de l’indice du minimum entre les indices début et fin du tableau reçu.
		\Module{positionMin}{tab\In~: tableau[1 à n] d’entiers, début, fin~: entiers}{entier}
			\Decl indiceMin, i~: entiers
			\Let indiceMin \Gets début
			\For{i \K{de} début+1 \K{à} fin}
				\If{tab[i] < tab[indiceMin]}
					\Decl indiceMin \Gets i
				\EndIf
			\EndFor
			\Return indiceMin
		\EndModule
	\end{pseudo}
	}
\end{frame}

\begin{frame}{Tri par sélection des minima successifs~: algorithme}
	\cadre{
	\begin{pseudo}
		\LComment {Échange le contenu de 2 variables.}
		\Module{swap}{a\InOut, b\InOut~: entiers}{}
			\Decl aux~: entiers
			\Let aux \Gets a
			\Let a \Gets b
			\Let b \Gets aux
		\EndModule
	\end{pseudo}
	}
\end{frame}

\subsection{Tri bulle}

\begin{frame}{Tri bulle}
	Il s’agit d’un tri par \textbf{permutations} ayant pour but d’amener à
	chaque étape à la «~surface~» du sous-tableau non trié (on entend par
	là l’élément d’indice minimum) la valeur la plus petite, appelée la
	\textbf{bulle}. 
	
	\bigskip
	
	La caractéristique de cette méthode est que les
	comparaisons ne se font qu’entre éléments consécutifs du tableau.
\end{frame}

\begin{frame}{Tri bulle}
	\begin{itemize}
		\item
		Prenons pour exemple un tableau de taille 14. 
		\item
		En partant de la fin du tableau, on le parcourt vers la 
		gauche en comparant chaque couple de valeurs consécutives. 
		\item
		Quand deux valeurs sont dans le désordre, on les permute. 
		\item
		Le premier parcours s’achève lorsqu’on arrive à l’élément
		d’indice 1 qui contient alors la «~bulle~»,
		c'est-à-dire la plus petite valeur du tableau, soit 1~:
	\end{itemize}
\end{frame}

\begin{frame}{Tri bulle}
	\begin{center}
	\begin{tabular}{*{14}{>{\centering\sffamily\itshape\arraybackslash}m{0.4cm}}}
		 1 &
		 2 &
		 3 &
		 4 &
		 5 &
		 6 &
		 7 &
		 8 &
		 9 &
		 10 &
		 11 &
		 12 &
		 13 &
		 14
		\\
	\end{tabular}
	\end{center}

	\begin{center}
	\begin{tabular}{|*{14}{>{\centering\arraybackslash}m{0.4cm}|}}
		\hline
		{10} &
		{ 5} &
		{12} &
		{15} &
		{ 4} &
		{ 8} &
		{ 1} &
		{ 7} &
		{12} &
		{11} &
		{ 3} &
		{ 6} &
		{ 5} &
		{\cellcolor{gray!25}4}\\\hline
	\end{tabular}
	\end{center}
	
	
	\begin{center}
	\begin{tabular}{|*{14}{>{\centering\arraybackslash}m{0.4cm}|}}
		\hline
		{10} &
		{ 5} &
		{12} &
		{15} &
		{ 4} &
		{ 8} &
		{ 1} &
		{ 7} &
		{12} &
		{11} &
		{ 3} &
		{ 6} &
		{\cellcolor{gray!25}4} &
		{ 5}\\\hline
	\end{tabular}
	\end{center}

	\begin{center}
	\begin{tabular}{|*{14}{>{\centering\arraybackslash}m{0.4cm}|}}
		\hline
		{10} &
		{ 5} &
		{12} &
		{15} &
		{ 4} &
		{ 8} &
		{ 1} &
		{ 7} &
		{12} &
		{11} &
		{ 3} &
		{\cellcolor{gray!25}4} &
		{ 6} &
		{ 5}\\\hline
	\end{tabular}
	\end{center}
\end{frame}

\begin{frame}{Tri bulle}
	\begin{center}
	\begin{tabular}{|*{14}{>{\centering\arraybackslash}m{0.4cm}|}}
		\hline
		{10} &
		{ 5} &
		{12} &
		{15} &
		{ 4} &
		{ 8} &
		{ 1} &
		{ 7} &
		{12} &
		{11} &
		{\cellcolor{gray!25}3} &
		{ 4} &
		{ 6} &
		{ 5}\\\hline
	\end{tabular}
	\end{center}

	\begin{center}
	\begin{tabular}{|*{14}{>{\centering\arraybackslash}m{0.4cm}|}}
		\hline
		{10} &
		{ 5} &
		{12} &
		{15} &
		{ 4} &
		{ 8} &
		{ 1} &
		{ 7} &
		{12} &
		{\cellcolor{gray!25}3} &
		{11} &
		{ 4} &
		{ 6} &
		{ 5}\\\hline
	\end{tabular}
	\end{center}

	\begin{center}
	\begin{tabular}{|*{14}{>{\centering\arraybackslash}m{0.4cm}|}}
		\hline
		{10} &
		{ 5} &
		{12} &
		{15} &
		{ 4} &
		{ 8} &
		{ 1} &
		{ 7} &
		{\cellcolor{gray!25}3} &
		{12} &
		{11} &
		{ 4} &
		{ 6} &
		{ 5}\\\hline
	\end{tabular}
	\end{center}
	
	\begin{center}
	\begin{tabular}{|*{14}{>{\centering\arraybackslash}m{0.4cm}|}}
		\hline
		{10} &
		{ 5} &
		{12} &
		{15} &
		{ 4} &
		{ 8} &
		{ 1} &
		{\cellcolor{gray!25}3} &
		{ 7} &
		{12} &
		{11} &
		{ 4} &
		{ 6} &
		{ 5}\\\hline
	\end{tabular}
	\end{center}

	\begin{center}
	\begin{tabular}{|*{14}{>{\centering\arraybackslash}m{0.4cm}|}}
		\hline
		{10} &
		{ 5} &
		{12} &
		{15} &
		{ 4} &
		{ 8} &
		{\cellcolor{gray!25}1} &
		{ 3} &
		{ 7} &
		{12} &
		{11} &
		{ 4} &
		{ 6} &
		{ 5}\\\hline
	\end{tabular}
	\end{center}

	\begin{center}
	\tablehead{}
	\begin{tabular}{|*{14}{>{\centering\arraybackslash}m{0.4cm}|}}
		\hline
		{10} &
		{ 5} &
		{12} &
		{15} &
		{ 4} &
		{\cellcolor{gray!25}1} &
		{ 8} &
		{ 3} &
		{ 7} &
		{12} &
		{11} &
		{ 4} &
		{ 6} &
		{ 5}\\\hline
	\end{tabular}
	\end{center}
\end{frame}

\begin{frame}{Tri bulle}
	\begin{center}
	\begin{tabular}{|*{14}{>{\centering\arraybackslash}m{0.4cm}|}}
		\hline
		{10} &
		{ 5} &
		{12} &
		{15} &
		{\cellcolor{gray!25}1} &
		{ 4} &
		{ 8} &
		{ 3} &
		{ 7} &
		{12} &
		{11} &
		{ 4} &
		{ 6} &
		{ 5}\\\hline
	\end{tabular}
	\end{center}

	\begin{center}
	\begin{tabular}{|*{14}{>{\centering\arraybackslash}m{0.4cm}|}}
		\hline
		{10} &
		{ 5} &
		{12} &
		{\cellcolor{gray!25}1} &
		{15} &
		{ 4} &
		{ 8} &
		{ 3} &
		{ 7} &
		{12} &
		{11} &
		{ 4} &
		{ 6} &
		{ 5}\\\hline
	\end{tabular}
	\end{center}
	
	\begin{center}
	\begin{tabular}{|*{14}{>{\centering\arraybackslash}m{0.4cm}|}}
		\hline
		{10} &
		{ 5} &
		{\cellcolor{gray!25}1} &
		{12} &
		{15} &
		{ 4} &
		{ 8} &
		{ 3} &
		{ 7} &
		{12} &
		{11} &
		{ 4} &
		{ 6} &
		{ 5}\\\hline
	\end{tabular}
	\end{center}
	
	\begin{center}
	\begin{tabular}{|*{14}{>{\centering\arraybackslash}m{0.4cm}|}}
		\hline
		{10} &
		{\cellcolor{gray!25}1} &
		{ 5} &
		{12} &
		{15} &
		{ 4} &
		{ 8} &
		{ 3} &
		{ 7} &
		{12} &
		{11} &
		{ 4} &
		{ 6} &
		{ 5}\\\hline
	\end{tabular}
	\end{center}

	\begin{center}
	\begin{tabular}{|*{14}{>{\centering\arraybackslash}m{0.4cm}|}}
		\hline
		\cellcolor{gray!25}1 & 
		10 & 
		5 & 
		12 & 
		15 &
		4 &
		8 &
		3 &
		7 &
		12 &
		11 &
		4 &
		6 &
		5
		\\\hline
	\end{tabular}
	\end{center}

	Ceci achève le premier parcours. 
\end{frame}

\begin{frame}{Tri bulle}
	On recommence à présent le même	processus dans le sous-tableau 
	débutant au deuxième élément, ce qui donnera le contenu suivant~:

	\begin{center}
	\begin{tabular}{|*{14}{>{\centering\arraybackslash}m{0.4cm}|}}
		\hline
		1 & 3 & 10 & 5 & 12 & 15 & 4 & 8 & 4 & 7 & 12 & 11 & 5 & 6
		\\\hline
	\end{tabular}
	\end{center}
\end{frame}

\begin{frame}{Tri bulle}
	Comme pour le tri par recherche des minima successifs, à l’issue de
	l’étape $i$ de l’algorithme, on obtient un sous-tableau trié
	(entre les indices 1 et \textit{i }) et un tableau non encore trié
	(entre les indices $i + 1$ et $n$). 
	
	Le tri est donc	achevé à l’issue de l’étape $n - 1$. 
	
	Cependant, on notera que
	petit à petit, outre le déplacement de la bulle, les éléments plus
	petits évoluent progressivement vers la gauche, et les plus gros vers
	la droite, de sorte qu’il est fréquent que le nombre d’étapes effectif
	est inférieur au nombre d’étapes théorique.
\end{frame}

\begin{frame}{Tri bulle~: algorithme}
	Dans cet algorithme, la variable \code{indiceBulle} est à
	la fois le compteur d’étapes et aussi l’indice de l’élément récepteur
	de la bulle à la fin d’une étape donnée.

	\bigskip
	
	\cadre{
	\begin{pseudo}
	\LComment Trie le tableau reçu en paramètre (via un tri bulle).
		\Module{triBulle}{tab\InOut~: tableau[1 à n] d’entiers}{}
			\Decl indiceBulle, i~: entiers
			\For{indiceBulle \K{de} 1 \K{à} n}
				\For{i \K{de} n – 1 \K{à} indiceBulle \K{par} – 1}
					\If{tab[i] > tab[i + 1]}
						\Stmt swap( tab[i], tab[i + 1] )
						\RComment voir algorithme précédent
					\EndIf
				\EndFor
			\EndFor
		\EndModule
	\end{pseudo}
	}
\end{frame}

\subsection{Cas particuliers}

\begin{frame}{Tri partiel}
	Parfois, on n’est pas intéressé par un tri complet de l’ensemble mais on
	veut uniquement les «~$k$~» plus petites (ou plus grandes) valeurs. Le
	tri par recherche des minima et le tri bulle sont particulièrement
	adaptés à cette situation ; il suffit de les arrêter plus tôt. Par
	contre, le tri par insertion est inefficace car il ne permet pas un
	arrêt anticipé.
\end{frame}

\begin{frame}{Sélection des meilleurs}
	Une autre situation courante est la suivante~: un ensemble de valeurs
	sont traitées (lues, calculées, ...) et il ne faut garder que les
	$k$ plus petites (ou plus grandes).
	
	L'algorithme est plus simple à écrire si on utilise
	une position supplémentaire en fin de tableau. 
	
	Notons aussi qu’il faut
	tenir compte du cas où il y a moins de valeurs que le nombre de valeurs
	voulues ; c’est pourquoi on ajoute une variable indiquant le nombre
	exact de valeurs dans le tableau.
\end{frame}

\begin{frame}{Exemple~: les $k$ plus petites valeurs}

		On lit un ensemble de valeurs strictement positives (un 0 indique la fin
		de la lecture) et on ne garde que les $k$ plus petites valeurs.

		\bigskip
		
		\cadre{
		\begin{pseudo}
			\Module{meilleurs}{plusPetits\InOut~: tableau[1 à k + 1] d’entiers, nbValeurs\Out~: entier}{}
				\Decl val~: entier
				\Let nbValeurs \Gets 0
				\Read val
				\While{val ${\neq}$ 0}
					\Stmt insérer( val, plusPetits, nbValeurs )
					\Read val
				\EndWhile
			\EndModule
		\end{pseudo}
		}
\end{frame}

\begin{frame}{Exemple~: les $k$ plus petites valeurs}
	\cadre{
	\begin{pseudo}
		\Module{insérer}{val\In~: entier, plusPetits\InOut~: tableau[1 à k + 1] d’entiers, nbValeurs\InOut~: entier}{}
				\Decl i~: entiers
				\Let i \Gets nbValeurs
				\Read val
				\While{i > 0 ET val < plusPetits[i]}
					\Let plusPetits[i + 1] \Gets plusPetits[i]
					\Let i \Gets i - 1
				\EndWhile
				\Let plusPetits[i + 1] \Gets val
				\If{nbValeurs < k}
					\Let nbValeurs \Gets nbValeurs + 1
				\EndIf
			\EndModule
		\end{pseudo}
		}
\end{frame}

\subsection{Recherche dichotomique}

\begin{frame}{Recherche dichotomique}
	La recherche dichotomique a pour essence de réduire à
	chaque étape la taille de l’ensemble de recherche de moitié, jusqu’à ce
	qu’il ne reste qu’un seul élément dont la valeur devrait être celle
	recherchée, sauf bien entendu en cas d’inexistence de cette valeur dans
	l’ensemble de départ. 
	
	\bigskip
	
	Nous l’expliquons pour un tableau ordonné mais
	elle peut s’appliquer aussi avec des changements mineurs pour
	d'autres structures (par exemple la liste ordonnée que
	nous étudierons plus tard).
\end{frame}

\begin{frame}{Recherche dichotomique}
	Soit \code{val} la valeur recherchée dans une zone
	délimitée par les indices \code{indiceGauche} et
	\code{indiceDroit}. 
	
	\bigskip
	
	On commence par déterminer l’élément
	médian, c’est-à-dire celui qui se trouve « au milieu » de la zone de
	recherche ; son indice sera déterminé par la formule

	{\centering
	\code{indiceMedian}\textsf{$\leftarrow$}\code{(indiceGauche +	indiceDroit) DIV 2}}
\end{frame}

\begin{frame}{Recherche dichotomique}
	N.B.~: cet élément médian n'est pas tout à fait au milieu dans 
	le cas d'une zone contenant un nombre pair d'éléments.
	\begin{itemize}
		\item
		On compare alors \code{val}
		avec la valeur de cet élément médian ; il est possible qu’on ait trouvé
		la valeur cherchée; sinon, on partage la zone de recherche en deux
		parties~: 
		\begin{itemize}
			\item
			une qui \textbf{ne contient certainement pas}
			la valeur cherchée et 
			\item
			une qui \textbf{pourrait la
			contenir}. 
		\end{itemize}
		\item
		C’est cette deuxième partie qui 
		devient la nouvelle zone de recherche. 
		\item
		On réitère le processus jusqu’à ce que la valeur cherchée soit trouvée ou
		que la zone de recherche soit réduite à sa plus simple expression,
		c’est-à-dire un seul élément.
	\end{itemize}
\end{frame}

\begin{frame}{Recherche dichotomique}
	\cadre{
	\begin{pseudo}
		\scriptsize{
		\Block{déclarationInitialisation}
			\Decl indiceDroit, indiceGauche, indiceMédian~: entiers
			\Decl candidat~: T
			\Decl trouvé~: booléen
			\Empty
			\Let indiceGauche \Gets 1
			\Let indiceDroit \Gets n
			\Let trouvé \Gets faux
		\EndBlock
	}
	\end{pseudo}
	}
\end{frame}

\begin{frame}{Recherche dichotomique}
	\cadre{
		\begin{pseudo}
			\scriptsize{
			\Module{rechercheDichotomique}{tab\InOut~: tableau[1 à n] de T, valeur\In~: T, pos\Out~: entier}{booléen}
				\Stmt déclarationInitialisation
				\While{NON trouvé ET indiceGauche {${\leq}$} indiceDroit}
					\Let indiceMédian \Gets (indiceGauche + indiceDroit) DIV 2
					\Let candidat \Gets tab[indiceMédian]
					\Switch{}
						\Case{candidat = valeur}
							\Let trouvé \Gets vrai
						\Case{candidat < valeur}
							\Let indiceGauche \Gets indiceMédian + 1
							\RComment on garde la partie droite
						\Case{candidat > valeur}
							\Let indiceDroit \Gets indiceMédian – 1
							\RComment on garde la partie gauche
					\EndSwitch
				\EndWhile
				\Stmt testTrouvé
				\Return trouvé
			\EndModule
			}
		\end{pseudo}
		}
\end{frame}

\begin{frame}{Recherche dichotomique}
	\cadre{
	\begin{pseudo}
		\scriptsize{
		\Block{testTrouvé}
				\If{trouvé}
					\Let pos \Gets indiceMédian
				\Else
					\Let pos \Gets indiceGauche
					\RComment dans le cas où la valeur n’est pas trouvée,
					\Empty 
					\RComment on vérifiera que indiceGauche donne la valeur 
					\Empty
					\RComment où elle pourrait être insérée.
				\EndIf
		\EndBlock
	}
	\end{pseudo}
	}
\end{frame}

\begin{frame}{Recherche dichotomique~: Exemple de recherche fructueuse}

		Supposons que l’on recherche la valeur	\textbf{23} dans un
		tableau de 20 entiers. 
		
		La zone ombrée représente à chaque fois la partie de recherche, 
		qui est au départ le tableau dans son entièreté.
		
		\bigskip
		
		Au départ,
		\code{indiceGauche} vaut 1 et
		\code{indiceDroit}	vaut 20.
			
		\begin{center}
		\begin{tabular}{*{20}{>{\centering\sffamily\itshape\arraybackslash}m{0.4cm}}}
			 1 &
			 2 &
			 3 &
			 4 &
			 5 &
			 6 &
			 7 &
			 8 &
			 9 &
			 ~~10 &
			 ~~11 &
			 ~~12 &
			 ~~13 &
			 ~~14 &
			 ~~15 & 
			 ~~16 &
			 ~~17 &
			 ~~18 &
			 ~~19 &
			 ~~20
			 \\
		\end{tabular}
		\begin{tabular}{|*{20}{>{\centering\arraybackslash}m{0.4cm}|}}
			\hline
			{\cellcolor{gray!25} 1} &
			{\cellcolor{gray!25} 3} &
			{\cellcolor{gray!25} 5} &
			{\cellcolor{gray!25} 7} &
			{\cellcolor{gray!25} 7} &
			{\cellcolor{gray!25} 9} &
			{\cellcolor{gray!25} 9} &
			{\cellcolor{gray!25}10} &
			{\cellcolor{gray!25}10} &
			{\cellcolor{gray!25}15} &
			{\cellcolor{gray!25}16} &
			{\cellcolor{gray!25}20} &
			{\cellcolor{gray!25}23} &
			{\cellcolor{gray!25}23} &
			{\cellcolor{gray!25}25} &
			{\cellcolor{gray!25}28} &
			{\cellcolor{gray!25}28} &
			{\cellcolor{gray!25}28} &
			{\cellcolor{gray!25}29} &
			{\cellcolor{gray!25}29}\\\hline
		\end{tabular}
		\end{center}
\end{frame}

\begin{frame}{Recherche dichotomique~: Exemple de recherche fructueuse}
	
		\textit{Première étape}~:
		
		\code{indiceMedian} \textsf{←} 
		\code{(1 + 20) DIV	2}, c’est-à-dire 10. 
		
		Comme la valeur en
		position 10 est 15, la valeur cherchée doit se trouver à sa droite, et
		on prend comme nouvelle zone de recherche celle délimitée par
		\code{indiceGauche} qui vaut 11 et
		\code{indiceDroit} qui vaut 20.
		
		\bigskip
		
		\begin{center}
		\begin{tabular}{*{20}{>{\centering\sffamily\itshape\arraybackslash}m{0.4cm}}}
			 1 &
			 2 &
			 3 &
			 4 &
			 5 &
			 6 &
			 7 &
			 8 &
			 9 &
			 ~~10 &
			 ~~11 &
			 ~~12 &
			 ~~13 &
			 ~~14 &
			 ~~15 & 
			 ~~16 &
			 ~~17 &
			 ~~18 &
			 ~~19 &
			 ~~20
			 \\
		\end{tabular}
		\begin{tabular}{|*{20}{>{\centering\arraybackslash}m{0.4cm}|}}
			\hline
			\multicolumn{1}{|m{0.49700004cm}|}{ 1} &
			{  3} &
			{  5} &
			{  7} &
			{  7} &
			{  9} &
			{  9} &
			{ 10} &
			{ 10} &
			{ 15} &
			{\cellcolor{gray!25} 16} &
			{\cellcolor{gray!25} 20} &
			{\cellcolor{gray!25} 23} &
			{\cellcolor{gray!25} 23} &
			{\cellcolor{gray!25} 25} &
			{\cellcolor{gray!25} 28} &
			{\cellcolor{gray!25} 28} &
			{\cellcolor{gray!25} 28} &
			{\cellcolor{gray!25} 29} &
			{\cellcolor{gray!25} 29}\\\hline
		\end{tabular}
		\end{center}
\end{frame}

\begin{frame}{Recherche dichotomique~: Exemple de recherche fructueuse}
		\textit{Deuxième étape}~:
			
		\code{indiceMedian} \textsf{←} 
		\code{(11 + 20) DIV 2}, c’est-à-dire 15. 
		
		Comme on y trouve la
		valeur 25, on garde les éléments situées à la gauche de celui-ci ; la
		nouvelle zone de recherche est [11, 14].
		
		\begin{center}
		\begin{tabular}{*{20}{>{\centering\sffamily\itshape\arraybackslash}m{0.4cm}}}
			 1 &
			 2 &
			 3 &
			 4 &
			 5 &
			 6 &
			 7 &
			 8 &
			 9 &
			 ~~10 &
			 ~~11 &
			 ~~12 &
			 ~~13 &
			 ~~14 &
			 ~~15 & 
			 ~~16 &
			 ~~17 &
			 ~~18 &
			 ~~19 &
			 ~~20
			 \\
		\end{tabular}
		\begin{tabular}{|*{20}{>{\centering\arraybackslash}m{0.4cm}|}}
			\hline
			\multicolumn{1}{|m{0.49700004cm}|}{ 1} &
			{  3} &
			{  5} &
			{  7} &
			{  7} &
			{  9} &
			{  9} &
			{ 10} &
			{ 10} &
			{ 15} &
			{\cellcolor{gray!25} 16} &
			{\cellcolor{gray!25} 20} &
			{\cellcolor{gray!25} 23} &
			{\cellcolor{gray!25} 23} &
			{ 25} &
			{ 28} &
			{ 28} &
			{ 28} &
			{ 29} &
			{ 29}\\\hline
		\end{tabular}
		\end{center}
\end{frame}

\begin{frame}{Recherche dichotomique~: Exemple de recherche fructueuse}
		\textit{Troisième étape}~:
		
		\code{indiceMedian} \textsf{←} 
		\code{(11 + 14) DIV 2}, c’est-à-dire 12 où se trouve l’élément
		20. 
		
		La zone de recherche devient
		\code{indiceGauche} vaut 13 et
		\code{indiceDroit} vaut 14.

		\begin{center}
		\begin{tabular}{*{20}{>{\centering\sffamily\itshape\arraybackslash}m{0.4cm}}}
			 1 &
			 2 &
			 3 &
			 4 &
			 5 &
			 6 &
			 7 &
			 8 &
			 9 &
			 ~~10 &
			 ~~11 &
			 ~~12 &
			 ~~13 &
			 ~~14 &
			 ~~15 & 
			 ~~16 &
			 ~~17 &
			 ~~18 &
			 ~~19 &
			 ~~20
			 \\
		\end{tabular}
		\begin{tabular}{|*{20}{>{\centering\arraybackslash}m{0.4cm}|}}
			\hline
			\multicolumn{1}{|m{0.49700004cm}|}{ 1} &
			{  3} &
			{  5} &
			{  7} &
			{  7} &
			{  9} &
			{  9} &
			{ 10} &
			{ 10} &
			{ 15} &
			{ 16} &
			{ 20} &
			{\cellcolor{gray!25} 23} &
			{\cellcolor{gray!25} 23} &
			{ 25} &
			{ 28} &
			{ 28} &
			{ 28} &
			{ 29} &
			{ 29}\\\hline
		\end{tabular}
		\end{center}
\end{frame}

\begin{frame}{Recherche dichotomique~: Exemple de recherche fructueuse}
		\textit{Quatrième étape~:}
		
		\code{indiceMedian} \textsf{←} 
		\code{(13 + 14) DIV 2}, c’est-à-dire 13 où se trouve la valeur
		cherchée ; la recherche est terminée.
\end{frame}

\begin{frame}{Recherche dichotomique~: Exemple de recherche infructueuse}
	Recherchons à présent la valeur \textbf{8}. 

	Les étapes de la recherche vont donner successivement
		
		\begin{center}
		\begin{tabular}{*{20}{>{\centering\sffamily\itshape\arraybackslash}m{0.4cm}}}
			 1 &
			 2 &
			 3 &
			 4 &
			 5 &
			 6 &
			 7 &
			 8 &
			 9 &
			 ~~10 &
			 ~~11 &
			 ~~12 &
			 ~~13 &
			 ~~14 &
			 ~~15 & 
			 ~~16 &
			 ~~17 &
			 ~~18 &
			 ~~19 &
			 ~~20
			 \\
		\end{tabular}
		\begin{tabular}{|*{20}{>{\centering\arraybackslash}m{0.4cm}|}}
			\hline
			\multicolumn{1}{|m{0.4cm}|}{\cellcolor{gray!25} 1} &
			{\cellcolor{gray!25}  3} &
			{\cellcolor{gray!25}  5} &
			{\cellcolor{gray!25}  7} &
			{\cellcolor{gray!25}  7} &
			{\cellcolor{gray!25}  9} &
			{\cellcolor{gray!25}  9} &
			{\cellcolor{gray!25} 10} &
			{\cellcolor{gray!25} 10} &
			{\cellcolor{gray!25} 15} &
			{\cellcolor{gray!25} 16} &
			{\cellcolor{gray!25} 20} &
			{\cellcolor{gray!25} 23} &
			{\cellcolor{gray!25} 23} &
			{\cellcolor{gray!25} 25} &
			{\cellcolor{gray!25} 28} &
			{\cellcolor{gray!25} 28} &
			{\cellcolor{gray!25} 28} &
			{\cellcolor{gray!25} 29} &
			{\cellcolor{gray!25} 29}\\\hline
		\end{tabular}
		\end{center}

		\bigskip
		
		\begin{center}
		\begin{tabular}{*{20}{>{\centering\sffamily\itshape\arraybackslash}m{0.4cm}}}
			 1 &
			 2 &
			 3 &
			 4 &
			 5 &
			 6 &
			 7 &
			 8 &
			 9 &
			 ~~10 &
			 ~~11 &
			 ~~12 &
			 ~~13 &
			 ~~14 &
			 ~~15 & 
			 ~~16 &
			 ~~17 &
			 ~~18 &
			 ~~19 &
			 ~~20
			 \\
		\end{tabular}
		\begin{tabular}{|*{20}{>{\centering\arraybackslash}m{0.4cm}|}}
			\hline
			\multicolumn{1}{|m{0.4cm}|}{\cellcolor{gray!25} 1} &
			{\cellcolor{gray!25}  3} &
			{\cellcolor{gray!25}  5} &
			{\cellcolor{gray!25}  7} &
			{\cellcolor{gray!25}  7} &
			{\cellcolor{gray!25}  9} &
			{\cellcolor{gray!25}  9} &
			{\cellcolor{gray!25} 10} &
			{\cellcolor{gray!25} 10} &
			{ 15} &
			{ 16} &
			{ 20} &
			{ 23} &
			{ 23} &
			{ 25} &
			{ 28} &
			{ 28} &
			{ 28} &
			{ 29} &
			{ 29}\\\hline
		\end{tabular}
		\end{center}

\end{frame}

\begin{frame}{Recherche dichotomique~: Exemple de recherche infructueuse}
		
		\begin{center}
		\begin{tabular}{*{20}{>{\centering\sffamily\itshape\arraybackslash}m{0.4cm}}}
			 1 &
			 2 &
			 3 &
			 4 &
			 5 &
			 6 &
			 7 &
			 8 &
			 9 &
			 ~~10 &
			 ~~11 &
			 ~~12 &
			 ~~13 &
			 ~~14 &
			 ~~15 & 
			 ~~16 &
			 ~~17 &
			 ~~18 &
			 ~~19 &
			 ~~20
			 \\
		\end{tabular}
		\begin{tabular}{|*{20}{>{\centering\arraybackslash}m{0.4cm}|}}
			\hline
			\multicolumn{1}{|m{0.4cm}|}{ 1} &
			{  3} &
			{  5} &
			{  7} &
			{  7} &
			{\cellcolor{gray!25}  9} &
			{\cellcolor{gray!25}  9} &
			{\cellcolor{gray!25} 10} &
			{\cellcolor{gray!25} 10} &
			{ 15} &
			{ 16} &
			{ 20} &
			{ 23} &
			{ 23} &
			{ 25} &
			{ 28} &
			{ 28} &
			{ 28} &
			{ 29} &
			{ 29}\\\hline
		\end{tabular}
		\end{center}

		\bigskip
		
		\begin{center}
		\begin{tabular}{*{20}{>{\centering\sffamily\itshape\arraybackslash}m{0.4cm}}}
			 1 &
			 2 &
			 3 &
			 4 &
			 5 &
			 6 &
			 7 &
			 8 &
			 9 &
			 ~~10 &
			 ~~11 &
			 ~~12 &
			 ~~13 &
			 ~~14 &
			 ~~15 & 
			 ~~16 &
			 ~~17 &
			 ~~18 &
			 ~~19 &
			 ~~20
			 \\
		\end{tabular}
		\begin{tabular}{|*{20}{>{\centering\arraybackslash}m{0.4cm}|}}
			\hline
			\multicolumn{1}{|m{0.49700004cm}|}{ 1} &
			{  3} &
			{  5} &
			{  7} &
			{  7} &
			{\cellcolor{gray!25}  9} &
			{  9} &
			{ 10} &
			{ 10} &
			{ 15} &
			{ 16} &
			{ 20} &
			{ 23} &
			{ 23} &
			{ 25} &
			{ 28} &
			{ 28} &
			{ 28} &
			{ 29} &
			{ 29}\\\hline
		\end{tabular}
		\end{center}
		
		\bigskip

		Arrivé à ce stade, la zone de recherche s’est réduite à un seul élément.
		Ce n’est pas celui qu’on recherche mais c’est à cet endroit qu’il se
		serait trouvé ; c’est donc là qu’on pourra éventuellement
		l'insérer. Le paramètre de sortie prend la valeur 6.
\end{frame}