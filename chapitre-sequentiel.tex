\section{Algorithmes séquentiels}
%\leconwithtoc

\subsection{Fraction}

\begin{frame}{Exemple}
	Rédiger une marche à suivre détaillée qui explique 
	comment additionner deux fractions~:
	
	\bigskip
	
	\cadre{
		\begin{enumerate}
			\item Rechercher le dénominateur commun des deux fractions
			\item Mettre chaque fraction au même dénominateur
			\item Additionner les numérateurs des deux fractions,
			 ce qui donne le numérateur de la somme
			\item Simplifier la fraction obtenue
		\end{enumerate}
	}
		
	Encore très imprécis. 
	
	Un algorithme proche d'un langage
	de programmation ne devrait mentionner 
	que les opérations élémentaires
	de calcul telles que $+$, -, $*$, $/$.
\end{frame}

\begin{frame}{Exemple plus proche d’un
		programme écrit dans un langage compréhensible par l’ordinateur}
	\cadre{
		\begin{enumerate}
		\item Prendre connaissance du premier numérateur et du premier dénominateur ;
		\item Prendre connaissance du second numérateur et du second dénominateur ;
		\item Multiplier les deux dénominateurs pour obtenir le dénominateur commun ;
		\item Multiplier le premier numérateur par le second dénominateur
			et le second numérateur par le premier dénominateur ;
		\item Additionner ces deux produits pour obtenir le numérateur du résultat ;
		\item Communiquer ce résultat ainsi que le dénominateur commun.
		\end{enumerate}
		}
\end{frame}

\begin{frame}{Langage}
	français~: 
	\begin{itemize}
	\item 
		peut toujours être
		utilisé dans la formalisation d’une première 
		approche de la résolution
		d’un problème.
	
	\item
		trop grande richesse de ce langage en
		comparaison à la pauvreté (mais aussi la rigueur) 
		du langage compris par la machine.
	
	\end{itemize}
\end{frame}

\subsection{Le pseudo-code}

\begin{frame}{Le pseudo-code}
	
	\bigskip
	Le \textbf{pseudo-code} ou \textbf{Langage de Description des
	Algorithmes} (LDA en abrégé) est un langage formel et symbolique
	utilisant :

	\begin{itemize}
	\item
		des \textbf{noms symboliques} destinés à représenter les objets sur
		lesquels s’effectuent des actions ;
	\item
		des \textbf{opérateurs symboliques} ou des mots-clés traduisant les
		opérations primitives exécutables par un exécutant donné ;
	\item
		des \textbf{structures de contrôle} types.
	\end{itemize}
\end{frame}

\subsection{Variables et types}

\begin{frame}{Variables et type}
	\begin{itemize}
		\item les opérations que l’ordinateur devra exécuter portent
			sur des éléments qui sont les \textbf{données} du problème~;
		\item lorsqu’on attribue un \textbf{nom} et un \textbf{type} 
			à ces données, on parle	alors de \textbf{variables}~;
		\item dans un algorithme, une variable conserve
			toujours son nom et son type, mais peut changer de \textbf{valeur}~;
		\item le \textbf{nom} d’une variable permet de la caractériser et de la
			reconnaitre~;
		\item le \textbf{type} d’une variable décrit la nature de son contenu.
	\end{itemize}
\end{frame}

\subsection{Les types autorisés}

\begin{frame}{Types autorisés}
	Dans un premier temps, les seuls \textbf{types} utilisés sont les
	suivants~:
	
	\bigskip
	
	\begin{tabular}{p{1.6cm}|p{11.5cm}}
	\raggedleft  \textbf{entier} &
	 pour les nombres entiers\\
	\raggedleft  \textbf{réel} &
	 pour les nombres réels\\
	\raggedleft  \textbf{caractère} &
	 pour les différentes lettres et caractères \\
	{~} & (par exemple ceux qui apparaissent sur un \\
	{~} & clavier~: ‘a’, ‘1’, ‘\#’, etc…)\\
	\raggedleft  \textbf{chaine} &
	pour les variables contenant plusieurs
	caractères\\
	{~} & ou aucun (la chaine vide) \\
	{~} &	(par exemple : "Bonjour", "Bonjour le monde",  \\
	{~} &	"a", "", etc.)
	\\
	\raggedleft  \textbf{booléen} &
	 les variables de ce type ne peuvent valoir que
	\textbf{vrai} ou \textbf{faux}\\
	\end{tabular}
\end{frame}

\begin{frame}{Exercice}
	Quel(s) type(s) de données utiliseriez-vous 
	pour représenter 
	\begin{itemize}
		\item une date du calendrier ?
		\item un moment dans la journée ?
		\item le prix d'un produit en grande surface ?
		\item votre nom ?
		\item vos initiales ?
		\item votre adresse ?
	\end{itemize}	
\end{frame}

\subsection{Déclaration de variables}

\begin{frame}{Déclaration de variables}
	La déclaration d’une variable est l’instruction 
	qui définit son nom et son type.
	\cadre{
		\begin{pseudo}
			\Declare{num1, num2}{entiers}
		\end{pseudo}
	}
	
	\bigskip
	
	L’ensemble des instructions de la forme

	\cadre{
	\begin{pseudo}
	\Declare{variable1, variable2,\ldots}{type}
	\end{pseudo}
	}

	forme la partie d’un algorithme nommée 
	\textbf{déclaration des variables}. 
\end{frame}

\begin{frame}{Déclaration de variables}
	La déclaration des informations apparaitra toujours 
	\begin{itemize}
	\item 
	en \textbf{début} d’algorithme, 
	\item
	ou dans un bloc annexe appelé 
	\textbf{dictionnaire des variables} 
	\item
	ou encore \textbf{dictionnaire des données}.
	\end{itemize}
\end{frame}

\begin{frame}{Déclaration de variables~: exemple}
	Pour l’algorithme des fractions, la déclaration des
	informations pourrait être la suivante :

	\cadre{
	\begin{pseudo}
	\Declare{num1, den1, num2, den2, numRes, denRes}{entiers}
	\end{pseudo}
	}

	avec la signification suivante :

	\begin{itemize}
	\item
		\textit{num1 (num2)}~: le numérateur 
		de la première (seconde) fraction ;
	\item
		\textit{den1 (den2)}~: le dénominateur 
		de la première (seconde) fraction ;
	\item
		\textit{numRes (denRes)}~:
		le numérateur (dénominateur) du résultat.
	\end{itemize}
\end{frame}

\begin{frame}{Déclaration de variables}
	\begin{itemize}
	\item
	Lors de la déclaration d’une variable, celle-ci n’a pas de
	valeur ! 
	\item
	Ce sera l’instruction d’\textbf{affectation} 
	qui va servir à donner un contenu aux variables
	déclarées. 
	\item
	En logique, nous n’envisageons pas d’\textit{affectation par
	défaut}
	\end{itemize}
\end{frame}

\subsection{Comment nommer correctement une variable ?}

\begin{frame}{Comment nommer correctement une variable ?}
	Un nom de variable
	\begin{itemize}
		\item est suffisamment court tout en restant explicite
		\item ne prête pas à confusion
		\item tient compte des limitations imposées par les
			langages de programmation
	\end{itemize}
\end{frame}

\begin{frame}{Comment nommer correctement une variable ?}
	Quelques règles et limitations traditionnelles dans les langages
	de programmation:

	\bigskip
	
	\begin{itemize}
	\item 
		Un nom de variable est généralement une suite de caractères
		alphanumériques d’un seul tenant (pas de caractères blancs) et ne
		commençant jamais par un chiffre. Ainsi \textit{x1} est
		correct mais non \textit{1x}. 
	\end{itemize}
\end{frame}

\begin{frame}{Comment nommer correctement une variable ?}
	\begin{itemize}
	\item 
		Pour donner un nom composé à une variable, on peut utiliser le 
		\textit{underscore} (par ex. premier\_numérateur) mais on
		déconseille d’utiliser le signe «~–~» qui est plutôt réservé à la
		soustraction. 
	
	\bigskip
	
	\item 
		Une alternative à l’utilisation du tiret bas pour l’écriture de noms de
		variables composés est la notation «~chameau~» (\textit{camelCase} en
		anglais), qui consiste à mettre une majuscule au début des mots
		(à partir du deuxième), par exemple
		\textit{premierNombre} ou
		\textit{dateNaissance}.
		\end{itemize}
	\end{frame}

\begin{frame}{Comment nommer correctement une variable ?}
	\begin{itemize}

	\item
		Les indices et exposants sont proscrits (par ex.
		\textit{$x_1$},
		\textit{$z_6$} ou
		\textit{$m^2$)}
		
	\bigskip
	
	\item
		On n'utilisera pas non plus les 
		mots-clés du langage/pseudo-code (tels que
		\textbf{\textsf{lire}}, 
		\textbf{\textsf{afficher}}, 
		\textbf{\textsf{pour}}…)
		
	\bigskip
	
	\item
		Certains langages n’autorisent pas les caractères accentués (tels que
		\textit{à, ç, ê, ø,} etc.) ou les lettres des alphabets non latins
		(tel que ${\Delta}$) mais d’autres oui ; certains font la distinction
		entre les minuscules et majuscules, d’autres non. En logique, nous
		admettons dans noms de variables les caractères accentués du français,
		par ex~: durée, intérêts, etc.
	\end{itemize}
\end{frame}

\begin{frame}{Exercices}
	Déclarer le(s) variable(s) permettant de représenter 
	\begin{itemize}
		\item la date d'anniversaire de quelqu'un.
		\item l'heure de début, l'heure de fin et	l'objet d'un rendez-vous.
	\end{itemize}
\end{frame}

\subsection{Opérateurs et expressions}

\begin{frame}{Opérateurs et expressions}
	Les opérateurs agissent sur les variables et les constantes pour former
	des \textbf{expressions}. 
	
	\bigskip
	
	Une expression est donc une combinaison
	\textbf{cohérente} de variables, de constantes et d’opérateurs,
	éventuellement accompagnés de parenthèses.
\end{frame}

\begin{frame}{Opérateurs arithmétiques élémentaires}
	Ce sont les opérateurs binaires bien connus :
	
	\bigskip
	
	\begin{tabular}{p{1.6cm}|p{11.5cm}}
	\raggedleft  \textit{+} & addition\\
	\raggedleft  \textit{-} & soustraction\\
	\raggedleft  \textit{*} & multiplication\\
	\raggedleft  \textit{/} & division réelle\\
	\raggedleft  \textit{DIV} & division entière\\
	\raggedleft  \textit{MOD} & reste de la division entière\\		
	\end{tabular}
	
	\bigskip
	
	Ils agissent sur des variables ou expressions à valeurs entières ou
	réelles. 
	
\end{frame}

\begin{frame}{Opérateurs arithmétiques élémentaires}
	\begin{itemize}
		\item
		Plusieurs opérateurs peuvent être utilisés pour former des
		expressions plus ou moins complexes, en tenant compte des règles de
		calcul habituelles, notamment la priorité de la multiplication et de la
		division sur l’addition et la soustraction. 
		
	\bigskip
	
		\item
		Il est aussi permis	d’utiliser des parenthèses.
	\end{itemize}
\end{frame}

\begin{frame}{Fonctions mathématiques complexes}
	\begin{itemize}
	\item
	L’élévation à la puissance sera notée \textit{**} ou
	\textit{\^{}} . 
	
	\bigskip
	
	\item
	La racine carrée d’une variable x s'écrira
	$\sqrt{x}$ \textit{.} 
	
	Attention, pour ce dernier, de veiller
	à ne l’utiliser qu’avec un radicant positif !

	\bigskip
	
	\item
	On se permettra d'utiliser les autres
	fonctions mathématiques sous leur forme la plus courante
	(exemples~:	$sin(x)$, $tan(x)$, $log(x)$, $exp(x)$, ...)
	\end{itemize}
\end{frame}

\begin{frame}{Fonctions mathématiques complexes}
	\textbf{Exemple} : 
	$(-b+\sqrt{(b\ast \ast 2-4\ast a\ast c)})/(2\ast a)$
	
	\bigskip
	
	Mais on peut aussi accepter la notation mathématique usuelle
	\begin{center}
	$\frac{-b+\sqrt{b^{2}-4\ast a\ast c}}{2\ast a}$ 
	\end{center}
	
	\bigskip

	Pourquoi ne pas avoir écrit «~\textit{4ac}~» et
	«~\textit{2a}~» ?
\end{frame}

\begin{frame}{Opérateurs de comparaison}

	Ils agissent sur des variables numériques ou des
	chaines et donnent un résultat booléen.

	\bigskip

	\begin{tabular}{p{1.6cm}|p{11.5cm}}
	\raggedleft  \textit{=} & égal\\
	\raggedleft  \textit{{\textless}{\textgreater}}
		ou \textit{${\neq}$} &  différent de\\
	\raggedleft  \textit{\textless} & (strictement) plus petit que\\
	\raggedleft  \textit{\textgreater} & (strictement) plus grand que\\
	\raggedleft  \textit{${\leq}$} & plus petit ou égal\\
	\raggedleft  \textit{${\geq}$} & plus grand ou égal\\
	\end{tabular}
	
	\bigskip

	Pour les chaines, c'est l’ordre alphabétique qui
	détermine le résultat (par exemple
	\textit{{\textquotedbl}milou{\textquotedbl} {\textless}
	{\textquotedbl}tintin{\textquotedbl}} est \textbf{vrai} de même que
	\textit{{\textquotedbl}assembleur{\textquotedbl}
	}\textit{${\leq}$}\textit{
	{\textquotedbl}java{\textquotedbl}})
	
\end{frame}
	
\begin{frame}{Opérateurs logiques}

	Ils agissent sur des expressions booléennes 
	pour donner un résultat du même type.
	
	\bigskip

	\begin{tabular}{p{1.6cm}|p{11.5cm}}
	\raggedleft  \textit{NON} & négation\\
	\raggedleft  \textit{ET} & conjonction logique\\
	\raggedleft  \textit{OU} & disjonction logique\\
	\end{tabular}
\end{frame}

\begin{frame}{Opérateurs logiques}
	\begin{itemize}
	\item
	\textbf{cond1 ET cond2} n’est vrai que lorsque
	les deux conditions sont vraies. 
	
	\item
	\textbf{cond1 OU cond2} est
	toujours vrai, sauf quand les deux conditions sont fausses.

	\item
	Parenthèser dans le cas de combinaisons de ET et de
	OU : 
	
	\textbf{(cond1 ET cond2) OU cond3} étant différent de
	
	\textbf{cond1 ET (cond2 OU cond3).} 
	
	\item
	En cas d'oubli de parenthèses,
	\textbf{\textit{ET} est prioritaire sur le \textit{OU}}.
	\end{itemize}
\end{frame}

\begin{frame}{Opérateurs logiques}

	Pour un booléen \textit{ok}~: 
	
	\begin{itemize}
	\item
	\textbf{ok = faux} est équivalent à \textbf{NON ok},
	\item
	\textbf{ok = vrai} est équivalent à \textbf{ok} et 
	\item
	\textbf{NON NON ok} est équivalent à \textbf{ok}.
	\end{itemize}
	Dans les trois cas, nous préconiserons la seconde écriture.
\end{frame}

\begin{frame}{Évaluation complète et court-circuitée}

	On définit deux modes d’évaluation des opérateurs \textit{ET}
	et \textit{OU} :

	\begin{itemize}
	\item{l’évaluation \textit{complète}}
	\item{l’évaluation \textit{court-circuitée}}
	\end{itemize}
	Dans le cadre de ce cours, nous opterons pour la deuxième 
	interprétation.
	
	\bigskip
	
	Exemple~: considérons	l’expression 
		{$n {\neq} 0 \ ET \  m/n > 10$}.
\end{frame}
		
\begin{frame}{Manipuler les chaines}
	Pour les chaines, nous allons introduire quelques notations
	qui vont nous permettre de les manipuler plus facilement.

	\begin{itemize}
	\item \textit{long(maChaine)}
		donne la taille (le nombre de caractères) de la chaine 
		\textit{maChaine}.
	\item \textit{car(maChaine,pos)}
		donne le caractère en position \textit{pos} 
		(à partir de 1) dans la chaine \textit{maChaine}.
	\item \textit{concat(maChaine1,maChaine2)}
		concatène les chaines \textit{maChaine1} 
		et \textit{maChaine2}.
		(ex: \textit{concat("Bon","jour")} donne \textit{"Bonjour"})
	\end{itemize}
\end{frame}

\begin{frame}{Manipuler les caractères}

		Introduisons également quelques notations pour les caractères.

		\begin{itemize}
		\item \textit{chaine(car)} transforme le caractère \textit{car} en une chaine de taille 1.
		\item \textit{estLettre(car)} est vrai si le caractère \textit{car} est une lettre
			(idem pour \textit{estChiffre}, 
			\textit{estMajuscule}, 
			\textit{estMinuscule}).
		\item \textit{majuscule(car)} donne la majuscule de la lettre \textit{car}
			(idem pour \textit{minuscule}).
		\item \textit{numLettre(car)} donne la position de \textit{car} dans l'alphabet (ex: \textit{numLettre('E')} donne 5, 
		idem pour \textit{numLettre('e')}).
		\item \textit{lettre(num)} l'inverse de la précédente (ex: \textit{lettre(4)} donne le caractère 'D').
	\end{itemize}
\end{frame}

\subsection{L’affectation d’une valeur à une variable}
\begin{frame}{Affectation externe}
	L’\textbf{affectation externe} est la primitive qui permet de recevoir de
	l’utilisateur, au moment où l'algorithme se déroule,
	une ou plusieurs valeur(s) et de les affecter à des variables en
	mémoire.

	\bigskip
	
	\cadre{
	\begin{pseudo}
	\Read liste\_de\_variables\_à\_lire
	\end{pseudo}
	}
\end{frame}

\begin{frame}{Affectation interne }
	On parle d’affectation interne lorsque la valeur d’une variable est
	«~calculée~» par l’exécutant de l’algorithme lui-même à partir de
	données qu’il connait déjà :

	\bigskip
	
	\cadre{
	\begin{pseudo}
	\Let nomVariable \Gets expression
	\end{pseudo}
	}
	
	\bigskip
	
	Par exemple~:

	\cadre{
	\begin{pseudo}
	\Let somme \Gets nombre1 + nombre2
	\Let denRes \Gets den1 * den2
	\Let cpt \Gets cpt + 1
	\Let delta \Gets b**2 – 4*a*c
	\Let test \Gets a < b \RComment pour une variable logique
	\Let maChaine \Gets "Bon"
	\Let uneChaine \Gets concat(maChaine, "jour")
	\end{pseudo}
	}
\end{frame}

\begin{frame}{Affectation}
	Il est de règle que le résultat de l’expression à droite du signe
	d’affectation ($\gets$) soit de
	même type que la variable à sa gauche. On tolère certaines exceptions~:
	\begin{itemize}
	\item
		\textit{varEntière}{ $\gets$ }\textit{varRéelle} : 
		dans ce cas le contenu de la variable sera la valeur \textbf{tronquée}
		de l’expression réelle. 
		
		Par exemple si «~\textit{nb}~» est
		une variable de type entier, son contenu après l’instruction
		«~\textit{nb}{ $\gets$ }\textit{$15/4$}~» 
		sera 3
	\item 
		\textit{varRéelle}{ $\gets$ }\textit{varEntière} :
		ici, il n'y a pas de perte de valeur.
	\item 
		\textit{varChaine}{ $\gets$ }\textit{varCaractère} : 
		
		équivalent à \textit{varChaine}{ $\gets$ }\textit{chaine(varCaractère)}.
		
		Le contraire n'est évidemment pas accepté.
	\end{itemize}
\end{frame}

\begin{frame}{Affectation}
	\begin{itemize}
	\item 
		Seules les variables déclarées peuvent être affectées, que ce soit par
		l’affectation externe ou interne!
	\item 
		Nous ne mélangerons pas la déclaration d’une variable et son
		affectation interne dans une même ligne de code, donc pas
		d’instructions hybrides du genre 
		\textsf{x}{ $\gets$ }\textit{2 : entier} ou encore 
		\textsf{x~: entier(0)}.
	\item 
		Pour l’affectation interne, toutes les variables apparaissant dans
		l'\textit{expression} doivent avoir été affectées
		préalablement. Le contraire provoquerait un arrêt de l’algorithme.
	\end{itemize}
	
\end{frame}

\subsection{Communication des résultats}
\begin{frame}{Communication des résultats}

	\cadre{
	\begin{pseudo}
	\Write \textit{expression} ou \textit{liste de variables séparées par des virgules}
	\end{pseudo}
	}

	\bigskip
	
	signifie que la valeur d’une expression (ou celles des différentes
	variables mentionnées) sera fournie à l’utilisateur (par exemple par un
	affichage à l’écran ou par impression sur listing via l’imprimante,
	etc\dots).
	
	\bigskip
	
	Comme pour l’affectation interne, on ne peut \code{afficher}
	que des expressions dont les variables qui la composent ont été 
	affectées préalablement.
\end{frame}

\subsection{Structure générale d’un algorithme}
\begin{frame}{Structure générale d’un algorithme}
	La traduction d’un algorithme en pseudo-code constituera le contenu d’un
	\code{module}. 
	
	\bigskip
	
	Un \code{module} contient donc la solution
	algorithmique d’un problème donné (ou d’une de ses parties). 
	
	\bigskip
	
	Sa structure générale sera la suivante :
	
	\bigskip
	
	\cadre{
	\begin{pseudo}
	\Module{nomDuModule}{}{}
		\Stmt \textit{déclaration des variables et constantes utilisées dans le module}
		\Stmt \textit{lecture des données}
		\Stmt \textit{instructions utilisant les données lues}
		\Stmt \textit{communication des résultats}
	\EndModule
	\end{pseudo}
	}		
	
	\bigskip
	
	Comme pour les variables, le nomDuModule devra être approprié au contenu.
\end{frame}

\begin{frame}{Fraction(suite)}
	Réécrivons l’algorithme d’addition de fractions décrit
	en début de chapitre~:

	\bigskip
	
	\cadre{
	\begin{pseudo}
	\Module{additionnerFractions}{}{}
		\Declare{num1, den1, num2, den2, numRes, denRes}{entiers}
		\Read num1, den1, num2, den2
		\Let denRes \Gets den1 * den2
		\Let numRes \Gets num1*den2 + num2*den1
		\Write numRes, "/", denRes
	\EndModule
	\end{pseudo}
	}
	
	\bigskip
	
	\textbf{Remarque~:}
	rappelons que la fraction affichée n'est sans doute pas simplifiée. 
	
	Nous n'avons pas encore tous les atouts suffisants pour réaliser 
	cela à ce niveau. Patience !
\end{frame}

\subsection{Commenter un algorithme}
\begin{frame}{Commenter un algorithme}
	On n’insistera jamais assez sur la nécessité de \textbf{documenter} un
	algorithme en y insérant des \textbf{commentaires} judicieux, clairs et
	suffisants ! 
	
	\bigskip
	
	Un \code{commentaire} est un texte placé dans
	l'algorithme et destiné à faciliter au maximum la
	compréhension d’un algorithme par le lecteur (parfois une autre
	personne, mais aussi souvent l'auteur qui se perd dans
	son propre texte lorsqu'il s'y replonge après une
	interruption). 
	
	\bigskip
	
	Ces commentaires (introduits par
	«~\textbf{//}~») seront bien entendu ignorés par
	l’exécutant de l’algorithme.
\end{frame}

\begin{frame}{Commenter fraction}
\cadre{
	\begin{pseudo}
	\LComment Lit les contenus de 2 fractions et affiche leur somme
	\Module{additionnerFractions}{}{}
		\Declare{num1, den1, num2, den2, numRes, denRes}{entiers}
		\Read num1, den1, num2, den2
		\Let denRes \Gets den1 * den2
			\RComment calcul du dénominateur
		\Let numRes \Gets num1*den2 + num2*den1
			\RComment calcul du numérateur
		\LComment la fraction n'est sans doute pas simplifiée
		\Write numRes, "/", denRes
	\EndModule
	\end{pseudo}
	}
\end{frame}

\begin{frame}{Commentaires}
	Noter qu’un excès de commentaires peut être aussi nuisible qu’un
	trop-peu pour la compréhension d’un algorithme. Par exemple, un choix
	judicieux de noms de variables peut s’avérer bien plus efficace que des
	commentaires superflus. 
	
	\bigskip
	
	Ainsi, l’instruction

	\cadre{
	\begin{pseudo}
	\Let nouveauCapital \Gets ancienCapital * (1 + taux / 100)
	\end{pseudo}
	}

	\bigskip
	
	dépourvue de commentaires est bien préférable aux lignes suivantes :

	\bigskip
	
	\cadre{
	\begin{pseudo}
	\Let c1 \Gets c0 * (1 + t / 100) 
	\LComment calcul du nouveau capital
	\LComment c1 est le nouveau capital, c0 est l’ancien capital, t est le taux
	\end{pseudo}
	}
	
	\bigskip
	
	Nous prendrons l'habitude de commenter chaque module en précisant ce qu'il fait.
\end{frame}

\begin{frame}{Constantes}

	Une \code{constante} est une information pour laquelle nom, type et
	valeur sont figés. 
	
	La liste des constantes utilisées dans un algorithme
	apparaitra dans la section déclaration des variables sous la forme
	suivante :

	\bigskip
	
	\cadre{
	\begin{pseudo}
	\Stmt \K{constante} PI = 3,14
	\Stmt \K{constante} ESI 
		= {\textquotedbl}École Supérieure d’Informatique{\textquotedbl}
	\end{pseudo}
	}

	\bigskip
	
	Il est inutile de spécifier leur type, celui-ci
	étant défini implicitement par la valeur de la constante.
\end{frame}

\begin{frame}{Énumération}

	Parfois, une variable ne peut prendre qu'un ensemble
	fixe et fini de valeurs. Par exemple une variable représentant une
	saison ne peut prendre que quatre valeurs (HIVER, PRINTEMPS, ÉTÉ,
	AUTOMNE). On va l'indiquer grâce à
	l'\code{enumeration} qui introduit un \textbf{nouveau type}
	de donnée.

	\bigskip
	
	\cadre{
	\begin{pseudo}
	\Stmt \K{énumération} Saison \{ HIVER, PRINTEMPS, ÉTÉ, AUTOMNE \}
	\end{pseudo}
	}

	\bigskip
	
	Il y a deux avantages à cela : une indication claire des possibilités de
	la variable lors de la déclaration et une lisibilité du code grâce à
	l'utilisation des valeurs explicites.
\end{frame}

\begin{frame}{Énumération~: exemple}
	Par exemple, 
		
	\cadre{
		\begin{pseudo}
			\LComment Lit une saison et affiche sa particularité
			\Module{particularitéSaisonnière}{}{}
				\Decl uneSaison : Saison
				\Read uneSaison
				\RComment on lira la valeur HIVER ou PRINTEMPS ou ÉTÉ ou AUTOMNE
				\If{uneSaison = HIVER}
					\Write "il neige"
				\Else
					\If{uneSaison = PRINTEMPS}
						\Write "les fleurs poussent"
					\Else
						\If{uneSaison = ÉTÉ}
							\Write "le soleil brille"
						\Else
							\Write "les feuilles tombent"
						\EndIf
					\EndIf
				\EndIf
			\EndModule
		\end{pseudo}
	}
\end{frame}

\begin{frame}{Lien avec les entiers}
	Dans l'exemple ci dessus, on lit une Saison mais souvent,
	si on travaille avec les Mois par exemple,
	on disposera plutôt d'un entier. 
	
	Il faut pouvoir	convertir les valeurs. 
	
	Nous allons adopter la syntaxe suivante :

	\bigskip
	
	\cadre{
	\begin{pseudo}
	\Stmt Saison(3) 
		\RComment donne l'énumération de la saison numéro 3 (on commence à 1);
		\LComment donne ÉTÉ dans notre exemple.
	\Stmt position(uneSaison)
		\RComment donne l'entier associé à une saison;
		\LComment si on a lu HIVER comme valeur pour uneSaison, 
		\LComment donne la valeur 1.
	\end{pseudo}
	}
\end{frame}
