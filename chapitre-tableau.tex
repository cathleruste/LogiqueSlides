\section{Tableaux}
%\leconwithtoc

\subsection{Objectifs}

\begin{frame}{Objectifs}
	Dans ce chapitre nous étudions les tableaux, 
	une structure qui peut contenir 
	plusieurs exemplaires de données similaires.
\end{frame}

\subsection{Utilité des tableaux}

\begin{frame}{Exemple~: Statistiques de vente}
	Un gérant d’une entreprise commerciale souhaite connaitre l’impact d’une
		journée de promotion publicitaire sur la vente de dix de ses produits.
		Pour ce faire, les numéros de ces produits (numérotés de 1 à 10 pour
		simplifier) ainsi que les quantités vendues pendant cette journée de
		promotion sont encodés au fur et à mesure de leurs ventes. En fin de
		journée, le vendeur entrera la valeur 0 pour signaler la fin de
		l’introduction des données. Ensuite, les statistiques des ventes seront
		affichées.
\end{frame}

\begin{frame}{Exemple~: Statistiques de vente}
	La démarche générale se décompose en trois parties~:

	\begin{itemize}
	\item {
		le traitement de début de journée, qui consiste essentiellement à mettre
		les compteurs des quantités vendues pour chaque produit à 0}
		\item {
		le traitement itératif durant toute la journée~: au fur et à mesure des
		ventes, il convient de les enregistrer, c’est-à-dire d’ajouter au
		compteur des ventes d’un produit la quantité vendue de ce produit ; ce
		traitement itératif s’interrompra lorsque la valeur 0 sera introduite}
	\item {
		le traitement final, consistant à communiquer les valeurs des compteurs
		pour chaque produit.}
	\end{itemize}
\end{frame}

\begin{frame}{Exemple~: Statistiques de vente}
	\cadre{
	\begin{pseudo}
	\LComment Calcule et affiche la quantité vendue de 10 produits.
	\Module{statistiquesVentesAvecTableau}{}{}
		\Stmt déclaration
		\Stmt initialisation
		\Stmt lecture
		\Stmt affichage
	\EndModule
\end{pseudo}
	}
\end{frame}

\begin{frame}{Exemple~: Statistiques de vente}
	\cadre{
	\begin{pseudo}
		\scriptsize{
		\Block{lecture}
		\Decl cpt1, cpt2, cpt3, cpt4, cpt5~: entiers
		\Decl cpt6, cpt7, cpt8, cpt9, cpt10~: entiers
		\Decl numéroProduit, quantité~: entiers
		\EndBlock
		}
	\end{pseudo}
	}
\end{frame}

\begin{frame}{Exemple~: Statistiques de vente}
	\cadre{
	\begin{pseudo}
		\scriptsize{
		\Block{initialisation}
		\Empty
		\Let cpt1 \Gets 0
		\Let cpt2 \Gets 0
		\Let cpt3 \Gets 0
		\Let cpt4 \Gets 0
		\Let cpt5 \Gets 0
		\Let cpt6 \Gets 0
		\Let cpt7 \Gets 0
		\Let cpt8 \Gets 0
		\Let cpt9 \Gets 0
		\Let cpt10 \Gets 0
		\EndBlock
		}
	\end{pseudo}
	}
\end{frame}

\begin{frame}{Exemple~: Statistiques de vente}
	\cadre{
	\begin{pseudo}
		\scriptsize{
		\Block{lecture}
		\Write "Introduisez le numéro du produit~:"
		\Read numéroProduit
		\While{numéroProduit > 0}
			\Write "Introduisez la quantité vendue~:"
			\Read quantité
			\Switch{numéroProduit \K{vaut}}
				\Stmt 1~: cpt1 \Gets cpt1 + quantité
				\Stmt 2~: cpt2 \Gets cpt2 + quantité
				\Stmt 3~: cpt3 \Gets cpt3 + quantité
				\Stmt 4~: cpt4 \Gets cpt4 + quantité
				\Stmt 5~: cpt5 \Gets cpt5 + quantité
				\Stmt 6~: cpt6 \Gets cpt6 + quantité
				\Stmt 7~: cpt7 \Gets cpt7 + quantité
				\Stmt 8~: cpt8 \Gets cpt8 + quantité
				\Stmt 9~: cpt9 \Gets cpt9 + quantité
				\Stmt 10~:cpt10 \Gets cpt10 + quantité
			\EndSwitch
			\Write "Introduisez le numéro du produit~:"
			\Read numéroProduit
		\EndWhile
		\EndBlock
		}
	\end{pseudo}
	}
\end{frame}

\begin{frame}{Exemple~: Statistiques de vente}
	\cadre{
	\begin{pseudo}
		\scriptsize{
		\Block{affichage}
		\Write "quantité vendue de produit 1~:", cpt1
		\Write "quantité vendue de produit 2~:", cpt2
		\Write "quantité vendue de produit 3~:", cpt3
		\Write "quantité vendue de produit 4~:", cpt4
		\Write "quantité vendue de produit 5~:", cpt5
		\Write "quantité vendue de produit 6~:", cpt6
		\Write "quantité vendue de produit 7~:", cpt7
		\Write "quantité vendue de produit 8~:", cpt8
		\Write "quantité vendue de produit 9~:", cpt9
		\Write "quantité vendue de produit 10~:", cpt10
	\EndBlock
	}
	\end{pseudo}
	}
\end{frame}

\begin{frame}{Exemple~: Statistiques de vente}
		Au lieu d’avoir à manier dix compteurs distincts
	(\textit{cpt1}, \textit{cpt2}, etc.), nous allons
	envisager une seule «~grande~» variable \textit{cpt} 
	compartimentée en dix «~sous-variables~» qui se distingueront 
	les unes des autres par un indice~: \textit{cpt1} 
	deviendrait ainsi \textit{cpt[1]}, \textit{cpt2} 
	deviendrait	\textit{cpt[2]}, et ainsi de suite jusqu’à
	\textit{cpt10} qui deviendrait \textit{cpt[10]}.
\end{frame}

%==============
%\begin{comment}

\begin{frame}{Exemple~: Statistiques de vente}

		\begin{center}
		\begin{tabular}{m{0.588cm}*{10}{m{1.05cm}}}
			~ &
			\centering  \textit{cpt[1]} &
			\centering  \textit{cpt[2]} &
			\centering  \textit{cpt[3]} &
			\centering  \textit{cpt[4]} &
			\centering  \textit{cpt[5]} &
			\centering  \textit{cpt[6]} &
			\centering  \textit{cpt[7]} &
			\centering  \textit{cpt[8]} &
			\centering  \textit{cpt[9]} &
			\centering\arraybslash 
			\textit{cpt[10]}\\\hhline{~----------}
			\multicolumn{1}{m{0.588cm}|}{\centering 
			\textit{cpt}} & 
			\multicolumn{1}{m{1.05cm}|}{~} &
			\multicolumn{1}{m{1.05cm}|}{~} &
			\multicolumn{1}{m{1.05cm}|}{~} &
			\multicolumn{1}{m{1.05cm}|}{~} &
			\multicolumn{1}{m{1.05cm}|}{~} &
			\multicolumn{1}{m{1.05cm}|}{~} &
			\multicolumn{1}{m{1.05cm}|}{~} &
			\multicolumn{1}{m{1.05cm}|}{~} &
			\multicolumn{1}{m{1.05cm}|}{~} &
			\multicolumn{1}{m{1.05cm}|}{~
			}\\\hhline{~----------}
		\end{tabular}
	\end{center}
		
	Un des intérêts de cette notation est la possibilité de faire apparaitre
	une variable entre les crochets, par exemple
	\textit{cpt[i]}, ce qui permet une grande économie de lignes
	de code.
\end{frame}

%\end{comment}
%============

\begin{frame}{Exemple~: Statistiques de vente}
	Voici la version avec tableau.

	\cadre{
	\begin{pseudo}
	\LComment Calcule et affiche la quantité vendue de 10 produits.
	\Module{statistiquesVentesAvecTableau}{}{}
		\Stmt déclaration
		\Stmt initialisation
		\Stmt lecture
		\Stmt affichage
	\EndModule
	\end{pseudo}
	}
\end{frame}

\begin{frame}{Exemple~: Statistiques de vente}
	\cadre{
	\begin{pseudo}
	\Block{déclaration}
		\Decl cpt~: \K{tableau} [1 à 10] d’entiers
		\Decl i, numéroProduit, quantité~: entiers
	\EndBlock
	\end{pseudo}
	}
\end{frame}

\begin{frame}{Exemple~: Statistiques de vente}
	\cadre{
	\begin{pseudo}
		\Block{initialisation}
		\For{i \K{de} 1 \K{à} 10}
			\Let cpt[i] \Gets 0
		\EndFor
		\EndBlock
	\end{pseudo}
	}
\end{frame}

\begin{frame}{Exemple~: Statistiques de vente}
	\cadre{
	\begin{pseudo}
		\scriptsize{
		\Block{lecture}
		\Write "Introduisez le numéro du produit~:"
		\Read numéroProduit
		\Empty
		\While{numéroProduit > 0}
		\Empty
			\Write "Introduisez la quantité vendue~:"
			\Read quantité
			\Empty
			\Let cpt[numéroProduit] \Gets cpt[numéroProduit] + quantité
			\Empty
			\Write "Introduisez le numéro du produit~:"
			\Read numéroProduit
			\Empty
		\EndWhile
		\EndBlock
		}
	\end{pseudo}
	}
\end{frame}

\begin{frame}{Exemple~: Statistiques de vente}
	\cadre{
	\begin{pseudo}
		\Block{affichage}
		\For{i \K{de} 1 \K{à} 10}
			\Write "quantité vendue de produit ", i, ": ", cpt[i]
		\EndFor
		\EndBlock
	\end{pseudo}
	}
\end{frame}

\subsection{Définitions}

\begin{frame}{définitions}
	Un \textbf{tableau} est une suite d’éléments de même type 
	portant tous le même nom mais se distinguant 
	les uns des autres par un indice.

	L’\textbf{indice} est un entier 
	donnant la position d’un élément dans la suite. 
	Cet indice varie entre la position du premier élément 
	et la position du dernier élément, 
	ces positions correspondant aux bornes de l’indice.
	Notons qu'il n'y a pas de «~trou~»~: 
	tous les éléments existent entre le premier et le dernier indice.

\end{frame}

\begin{frame}{définitions}
	La \textbf{taille} d’un tableau 
	est le nombre (strictement positif) de ses éléments.
	Attention ! la taille d’un tableau ne peut pas être modifiée pendant
	son utilisation.

	Souvent on utilise un tableau plus grand que
	le nombre utile de ses éléments. 
	Seule une partie du tableau est utilisée. 
	On parle alors de taille \textbf{physique}
	(la taille maximale du tableau) 
	et de taille \textbf{logique}
	(le nombre d'éléments effectivement utilisés)
\end{frame}

\subsection{Notations}

\begin{frame}{notations}
	Pour déclarer un tableau, on écrit~:

	\cadre{
	\begin{pseudo}
	\Decl nomTableau~: \K{tableau} [borneMin à borneMax] de TypeElément
	\end{pseudo}
	}
	
	où \textit{TypeElément} est le type des éléments que l’on
	trouvera dans le tableau. 
	
	\bigskip
	
	Les éléments sont d’un des types élémentaires
	vus précédemment (entier, réel, booléen, chaine, caractère) ou encore
	des variables structurées. 
	
	\bigskip
	
	À ce propos, remarquons aussi
	qu'un tableau peut être un champ d'une structure. 
	D'autres possibilités apparaitront lors de l'étude de
	l'orienté objet.

\end{frame}

\begin{frame}{notations}
	Les bornes apparaissant dans la déclaration sont 
	\begin{itemize}
	\item
	des constantes ou 
	\item
	des paramètres ayant une valeur connue lors de la déclaration. 
	\end{itemize}
\end{frame}

\begin{frame}{notations}
	Une fois un
	tableau déclaré, seuls les éléments d’indice compris entre
	\textit{borneMin} et \textit{borneMax} peuvent
	être utilisés. 
	
	\bigskip
	
	Par exemple, si on déclare~:

	\cadre{
	\begin{pseudo}
	\Decl tabEntiers~: \K{tableau} [1 à 100] d’entiers
	\end{pseudo}
	}
	
	\bigskip
	
	Il est interdit d’utiliser \textit{tabEntiers[0]} ou
	\textit{tabEntiers[101]}. De plus, chaque élément
	\textit{tabEntiers[i]} (avec $1 \leq i \leq 100$) doit
	être manié avec la même précaution qu’une variable simple, c’est-à-dire
	qu’on ne peut utiliser un élément du tableau qui n’aurait pas été
	préalablement affecté ou initialisé.
\end{frame}

\begin{frame}{notations}
	Il n'est pas interdit de prendre 0 pour la borne
	inférieure ou même d'utiliser des bornes négatives
	(par exemple~: \textit{tabTempératures~: 
	\textbf{tableau} [-20 à 50] de réels}).

	\bigskip
	
	En Java, un tableau est défini par sa taille $n$ et les bornes sont
	automatiquement $0$ et $n-1$. Ce n'est pas le cas en
	Logique où on a plus de liberté dans le choix des bornes.
\end{frame}

\subsection{Tableau statique vs tableau dynamique}

\begin{frame}{Tableau statique vs tableau dynamique}
	Les tableaux étudiés jusqu'ici sont dit \textbf{statiques}.
	
	\begin{itemize}
	\item
	Un tableau est dit \textbf{statique} si sa taille 
	est connue lors de l’écriture du programme
	\item
	Un tableau est dit \textbf{dynamique}
	si sa taille n’est connue qu’à l’exécution du programme (elle est
	calculée, donnée par l’utilisateur, lue dans un fichier de
	configuration, fournie par une autre partie du programme, \dots)
	\end{itemize}
\end{frame}

\begin{frame}{Tableau statique vs tableau dynamique~: exemple}
	\begin{itemize}
	\item
		Reprenons l’exemple ci-dessus. 
		S’il y a toujours exactement 10 produits, 
		alors on peut utiliser un tableau statique. 
	\item
		Dans la pratique, il est probable que ce nombre puisse évoluer 
		au cours de l’histoire de l’entreprise.
		
		On peut au moins demander au gérant le nombre d'articles dans son stock.
		
		On utilisera alors un tableau dynamique.
	\item 
		Si on désire stocker le chiffre d’affaire d’une entreprise 
		pour une année donnée, mois par mois, un tableau de
		taille 12 est suffisant (tableau statique)
	\end{itemize}
\end{frame}

\begin{frame}{Tableau statique vs tableau dynamique~: concrètement, comment choisir~?}
	Le choix dépend du langage.

	Certains langages (c’est le cas de Cobol) ne permettent 
	que des tableaux statiques. Dans ce cas, il faudra souvent
	imposer une \textbf{taille maximale} au tableau. 
	Il sera également nécessaire d’utiliser une variable 
	pour retenir la taille utile (ou logique) du tableau, 
	à savoir la partie du tableau réellement utilisée.
	Bien sûr cette solution entraine souvent une perte de place mémoire due
	à une réservation inutilement grande (ou, pire, une saturation du
	tableau qui n’aura pas été prévu assez grand).
	
	\bigskip
	
	D'autres, comme Java, n'ont que des tableaux dynamiques. La taille ne doit 
	pas forcément être connue à la compilation, mais doit être connue à
	l'exécution. Vous verrez les détails dans votre cours de Java.
\end{frame}

\begin{frame}{Tableau statique vs tableau dynamique~: exemple statique}
	Reprenons encore l’exemple de la vente de produits. 
	
	Si on ne dispose pas de tableau dynamique, on peut réserver
	un tableau de taille 1000 par exemple (si on est sûr qu’on ne vendra
	pas plus de 1000 produits différents) et ajouter une variable entière
	(\textit{nbProduits}) pour retenir 
	le nombre exact de produits différents actuellement en vente.
\end{frame}

\begin{frame}{Tableau statique vs tableau dynamique~: exemple statique}
	\cadre{
	\begin{pseudo}
	\LComment Calcule et affiche la quantité vendue de x produits.
	\Module{statistiquesVentes}{}{}
		\Stmt déclaration
		\Stmt initialisation
		\Stmt lecture
		\Stmt affichage
	\EndModule
	\end{pseudo}
	}
\end{frame}

\begin{frame}{Exemple~: Statistiques de vente}
	\cadre{
	\begin{pseudo}
		\Block{déclaration}
		\Decl cpt~: \K{tableau} [1 à 1000] d’entiers
		\Decl i, numéroProduit, quantité~: entiers
		\Decl nbArticles~: entier
		\EndBlock
	\end{pseudo}
	}
\end{frame}

\begin{frame}{Exemple~: Statistiques de vente}
	\cadre{
	\begin{pseudo}
		\Block{initialisation}
		\Read nbArticles
		\Empty
		\For{i \K{de} 1 \K{à} nbArticles}
			\Let cpt[i] \Gets 0
		\EndFor
		\EndBlock
	\end{pseudo}
	}
\end{frame}

\begin{frame}{Exemple~: Statistiques de vente}
	\cadre{
	\begin{pseudo}
		\scriptsize{
		\Block{lecture}
		\Write "Introduisez le numéro du produit~:"
		\Read numéroProduit
		\Empty
		\While{numéroProduit > 0 ET numéroProduit < nbArticles}
		\Empty
			\Write "Introduisez la quantité vendue~:"
			\Read quantité
			\Empty
			\Let cpt[numéroProduit] \Gets cpt[numéroProduit] + quantité
			\Empty
			\Write "Introduisez le numéro du produit~:"
			\Read numéroProduit
		\EndWhile
		\EndBlock
		}
	\end{pseudo}
	}
\end{frame}

\begin{frame}{Exemple~: Statistiques de vente}
	\cadre{
	\begin{pseudo}
		\Block{affichage}
		\For{i \K{de} 1 \K{à} 10}
			\Write "quantité vendue de produit ", i, ": ", cpt[i]
		\EndFor
		\EndBlock
	\end{pseudo}
	}
\end{frame}


\begin{frame}{Tableau statique vs tableau dynamique~: exemple dynamique}
Pour déclarer un tableau dynamique, on sépare la déclaration du tableau
	(où la taille n’est pas donnée) de sa création effective~(où on donne
	la taille)

	\cadre{
	\begin{pseudo}
	\Decl nomTableau~: \K{tableau} de TypeElément
	\LComment Le code peut déterminer ici les bornes
	\Let nomTableau \Gets \K{nouveau} \K{tableau} [ borneMin à borneMax ] de TypeElément
	\end{pseudo}
	}
	
	\bigskip

	où \textit{borneMin} et \textit{borneMax} sont des
	expressions entières quelconques.
\end{frame}

\begin{frame}{Tableau statique vs tableau dynamique~: exemple dynamique}
	\cadre{
	\begin{pseudo}
	\LComment Calcule et affiche la quantité vendue de 10 produits.
	\Module{statistiquesVentesAvecTableau}{}{}
		\Stmt déclaration
		\Stmt initialisation
		\Stmt lecture
		\Stmt affichage
	\EndModule
	\end{pseudo}
	}
\end{frame}

\begin{frame}{Exemple~: Statistiques de vente}
	\cadre{
	\begin{pseudo}
		\Block{déclaration}
		\Decl cpt~: \K{tableau} d’entiers
		\Decl i, numéroProduit, quantité~: entiers
		\Decl nbArticles~: entier
		\EndBlock
	\end{pseudo}
	}
\end{frame}

\begin{frame}{Exemple~: Statistiques de vente}
	\cadre{
	\begin{pseudo}
		\Block{initialisation}
		\Read nbArticles
		\Let cpt \Gets \K{nouveau} \K{tableau} [1 à nbArticles] d'entiers
		\Empty
		\For{i \K{de} 1 \K{à} nbArticles}
			\Let cpt[i] \Gets 0
		\EndFor
	\EndBlock
	\end{pseudo}
	}
\end{frame}

\begin{frame}{Exemple~: Statistiques de vente}
	\cadre{
	\begin{pseudo}
		\Block{lecture}
		\Write "Introduisez le numéro du produit~:"
		\Read numéroProduit
		\Empty
		\While{numéroProduit > 0 ET numéroProduit < nbArticles}
		\Empty
			\Write "Introduisez la quantité vendue~:"
			\Read quantité
			\Empty
			\Let cpt[numéroProduit] \Gets cpt[numéroProduit] + quantité
			\Empty
			\Write "Introduisez le numéro du produit~:"
			\Read numéroProduit
			\Empty
		\EndWhile
	\EndBlock
	\end{pseudo}
	}
\end{frame}

\begin{frame}{Exemple~: Statistiques de vente}
	\cadre{
	\begin{pseudo}
		\Block{affichage}
		\For{i \K{de} 1 \K{à} 10}
			\Write "quantité vendue de produit ", i, ": ", cpt[i]
		\EndFor
	\EndBlock
	\end{pseudo}
	}
\end{frame}

\subsection{Tableau et paramètres}

\begin{frame}{Tableau et paramètres}
	Un tableau peut être passé en paramètre à un module mais qu’en est-il de
	sa taille ? Il serait utile de pouvoir appeler le même module avec des
	tableaux de tailles différentes. Pour permettre cela, la taille du
	tableau reçu en paramètre est déclarée avec une variable (qui peut être
	considéré comme un paramètre entrant).
\end{frame}

\begin{frame}{Tableau et paramètres~: exemple}
	\cadre{
	\begin{pseudo}
	\Module{brol}{tab\In~: \K{tableau} [1 à n] d'entiers}{}
		\Write "J’ai reçu un tableau de ", n, " éléments".
	\EndModule
	\end{pseudo}
	}
	
	\bigskip

	Ce $n$ va prendre la taille précise du tableau
	(statique ou dynamique) utilisé à chaque appel et peut être utilisé
	dans le corps du module. Bien sûr il s’agit là de la taille physique du
	tableau.
\end{frame}

\begin{frame}{Tableau et paramètres~: exemple}
	Si une partie seulement du tableau doit être traitée, il
	convient de passer également la taille logique en paramètre.
	
	\cadre{
	\begin{pseudo}
	\Module{brol}{tab\In~: \K{tableau} [1 à n] d'entiers, tailleLogique~: entier}{}
		\Write "J’ai reçu un tableau rempli de ", tailleLogique, " éléments "
		\Write "sur ", n, " éléments au total."
	\EndModule
	\end{pseudo}
	}
\end{frame}

\begin{frame}{Tableau et paramètres~: exemple}
	Notons qu’on admet également qu’un module retourne un tableau. 
	
	\bigskip
	
	\cadre{
	\begin{pseudo}
	\LComment Crée un tableau statique d'entiers de taille 10, l'initialise à 0 et le retourne.
	\Module{créerTableau}{}{tableau [1 à 10] d'entiers}
		\Decl tab~: \K{tableau} [1 à 10] d'entiers
		\Decl i~: entier
		\For{i \K{de} 1 \K{à} 10}
			\Let tab[i] \Gets 0
		\EndFor
		\Return tab
	\EndModule
	\end{pseudo}
	}
\end{frame}


\begin{frame}{Tableau et paramètres~: exemple}
	\cadre{
	\begin{pseudo}	
	\Module{principalAppelTableau}{}{}
		\Decl entiers~: \K{tableau} [1 à 10] d'entiers
		\Decl i~: entier
		\Empty
		\Let entiers \Gets créerTableau()
		\Empty
		\For{i \K{de} 1 \K{à} 10}
			\Write entiers[i]
		\EndFor
	\EndModule
	\end{pseudo}
	}
\end{frame}

\begin{frame}{Tableau et paramètres~: exemple}
	Par contre, on ne peut pas lire ou afficher un tableau en une seule
	instruction; il faut des instructions de lecture ou
	d'affichage individuelles pour chacun de ses éléments.
\end{frame}

\subsection{Parcours d'un tableau à une dimension}

\begin{frame}{Parcours d'un tableau à une dimension}
	Soit le tableau statique \textit{tab} déclaré ainsi

	\bigskip

	\cadre{
	\begin{pseudo}
		\Decl tab~: \K{tableau} [1 à n] de T \RComment où \code{T} est un type quelconque
	\end{pseudo}
	}
	
	\bigskip

	ou soit le tableau dynamique \textit{tab} déclaré ainsi
	
	\bigskip

	\cadre{
	\begin{pseudo}
		\Decl tab~: \K{tableau} de T \RComment où \code{T} est un type quelconque
		\Let tab \Gets \K{nouveau} \K{tableau} [1 à n] de T
	\end{pseudo}
	}
\end{frame}

\begin{frame}{Parcours d'un tableau à une dimension}
	Envisageons d'abord le parcours complet
	et voyons ensuite les parcours avec arrêt prématuré.
\end{frame}

\begin{frame}{Parcours d'un tableau à une dimension~: parcours complet}
	Pour parcourir complètement un tableau, 
	on peut utiliser la boucle \K{pour}
	comme dans l'algorithme suivant
	où \og{}traiter\fg{} va dépendre du problème concret posé~:
	afficher, modifier, sommer, \dots
\end{frame}

\begin{frame}{Parcours d'un tableau à une dimension~: parcours complet}
	\cadre{
	\begin{pseudo}
		\LComment{Parcours complet d'un tableau via une boucle pour}
		\LComment Les déclarations sont omises pour ne pas alourdir les algorithmes.
		\Empty
		\For{i de 1 à n}
			\Stmt traiter tab[i]
		\EndFor
	\end{pseudo}
	}
\end{frame}

\begin{frame}{Parcours d'un tableau à une dimension~: parcours avec sortie prématurée}
	Parfois, on ne doit pas forcément parcourir le tableau jusqu'au bout
	mais on pourra s'arrêter prématurément si une certaine condition est remplie.
	
	\bigskip
	
	Par exemple~:
	\begin{itemize}
		\item on cherche la présence d'un élément et on vient de le trouver ;
		\item on vérifie qu'il n'y a pas de $0$ et on vient d'en trouver un.
	\end{itemize}
\end{frame}

\begin{frame}{Parcours d'un tableau à une dimension~: parcours avec sortie prématurée}
	La première étape est de transformer le \K{pour} en \K{tant que}
	ce qui donne l'algorithme 

	\cadre{
	\begin{pseudo}
		\LComment{Parcours complet d'un tableau via une boucle tant-que}
		\Empty
		\Let i \Gets 1
		\While{i $\le$ n}
			\Stmt traiter tab[i]
			\Let i \Gets i + 1
		\EndWhile
	\end{pseudo}
	}
\end{frame}

\begin{frame}{Parcours d'un tableau à une dimension~: parcours avec sortie prématurée}
	On peut à présent introduire le test d'arrêt.
	
	Une contrainte est qu'on voudra, à la fin de la boucle, savoir
	si oui ou non on s'est arrêté prématurément et, si c'est le cas,
	à quel indice.
	
	\bigskip

	Il existe essentiellement deux solutions, avec ou sans variable booléenne.
	En général, la première solution sera plus claire si le test est court.
\end{frame}

\begin{frame}{Parcours d'un tableau à une dimension~: parcours avec sortie prématurée}
	\cadre{
	\begin{pseudo}
		\LComment{Parcours partiel d'un tableau sans variable booléenne}
		\Let i \Gets 1
		\While{i $\le$ n ET \textit{test sur \code{tab[i]} dit que on continue}}
			\Let i \Gets i + 1
		\EndWhile
		\If{i $>$ n}
			\LComment on est arrivé au bout
		\Else
			\LComment arrêt prématuré à l'indice i.
		\EndIf
	\end{pseudo}
	}
\end{frame}

\begin{frame}{Parcours d'un tableau à une dimension~: parcours avec sortie prématurée}
	On pourrait inverser les deux branches du \K{si-sinon} en inversant le test
	mais attention à ne pas tester tab[i] car $i$ n'est peut-être pas valide.
	
	\cadre{
	\begin{pseudo}
		\LComment{Parcours partiel d'un tableau avec variable booléenne}
		\Let i \Gets 1
		\Let trouvé \Gets faux
		\While{i $\le$ n ET NON trouvé}
			\If{\textit{test sur \code{tab[i]} dit que on a trouvé}}
				\Let trouvé \Gets vrai
			\Else
				\Let i \Gets i + 1
			\EndIf
		\EndWhile
		\LComment tester le booléen pour savoir si arrêt prématuré.
	\end{pseudo}
	}
\end{frame}

\begin{frame}{Parcours d'un tableau à une dimension~: parcours avec sortie prématurée}
	Attention à bien choisir un nom de booléen adapté au problème
	et à l'initialiser à la bonne valeur. 
	
	\bigskip
	
	Par exemple, si la variable s'appelle \textit{continue}
	\begin{itemize}
		\item initialiser la variable à \textit{vrai} ;
		\item le test de la boucle est \textit{(... ET continue)} ;
		\item mettre la variable à \textit{faux} pour sortir de la boucle.
	\end{itemize}
\end{frame}